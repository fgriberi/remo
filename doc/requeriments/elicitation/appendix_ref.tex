\section{Referencias}
\label{appendix-ref}
[1] IEEE Recommended Practice for Software Requirements Specifications. Copyright © 1998 by the Institute of Electrical and Electronics Engineers, Inc. All rights reserved. Published 1998. Printed in the United States of America. ISBN 0-7381-0332-2. \\

[2] SOLID: ``Design Principles and Design Patterns'', Robert C. Martin. 
\textcolor{blue}{http://www.objectmentor.comresources/}\\
\textcolor{blue}{articles/Principles\_and\_Patterns.pdf} \\


[3] C++: Lenguaje de programación. \textcolor{blue}{http://www.cplusplus.com} \\

[4] G. Biset, D. Gutson, and M. Arroyo, “A framework for small distributed projects and a protein clusterer application”, 2009. \\

[5] G. Biset, D. Gutson, and M. Arroyo, “Fud: Design and implementation of a framework for small distributed applications”, 2009. \\

[6] B. Meyer, “Object-Oriented Software Construction”, Second Edition, Santa Barbara: Prentice Hall Professional Technical Reference, 1997. \\

[7] G. Booch, J. Rumbaugh, and I. Jacobson, “Unified Modeling Language User Guide”, Second Edition, 2005. \\

[8] RFC 2119. \textcolor{blue}{http://tools.ietf.org/html/rfc2119} \\

[9] GNU General Public License. \textcolor{blue}{http://www.gnu.org/licenses/} \\

[10] H. Curtis, N. Sue Barnes, A. Schnek and G. Flores, “Biología”, Editorial Médica Panamericana S.A, 2006, ISBN: 950-06-0423-X. \\

[11] B. Pierce, “Genética. Un enfoque conceptual”, Tercera Edición, Editorial médica panamericana S.A, ISBN: 978-84-9835-216-0. \\

[12] A. Blanco, “Química Biológica”, Séptima Edición, Editorial El Ateneo. \\

[13] $\Delta$(G): \textcolor{blue}{http://en.wikipedia.org/wiki/Gibbs\_free\_energy} \\

[14] FuD : \textcolor{blue}{http://code.google.com/p/fud/} \\

[15] fudepan-build: \textcolor{blue}{http://fudepan-build.googlecode.com} \\

[16] BLAST: \textcolor{blue}{http://blast.ncbi.nlm.nih.gov/Blast.cgi} \\

[17] FuDePAN: \textcolor{blue}{http://www.fudepan.org.ar/} \\

[18] Vinay S. Mahajan, Adam Drake and Jianzhu Chen, “Virus-specific host miRNAs: antiviral defenses or promoters of persistent infection?”. \\

[19] Man Lung YEUNG, Yamina BENNASSER, Shu Yun LE and Kuan Teh JEANG, “siRNA, miRNA and HIV: promises and challenges”. \\

[20] Gareth M. Jenkins and Edward C. Holmes, “The extent of codon usage bias in human RNA viruses and its evolutionary origin”, 2003. \\

[21] Comeron JM and Aguadé M. “An evaluation of measures of synonymous codon usage bias”, 1998. \\

[22] Haruhiko Siomi and Mikiko C. Siomi, “On the road to reading the RNA-interference code”. \\
