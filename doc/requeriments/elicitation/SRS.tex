\documentclass[12pt,a4paper,english,spanish]{article}

%package
\usepackage[T1]{fontenc}
\usepackage[utf8]{inputenc}
\pagestyle{headings}
\usepackage{babel}
\usepackage{xcolor}
\usepackage{amsmath}
\usepackage{amsfonts}
\usepackage{amssymb}
\usepackage[pdftex]{graphicx}
\usepackage[unicode=true] {hyperref}
\usepackage[toc, page, header]{appendix}

\makeatletter

%%%%%%%%%%%%%%%%%%%%%%%%%%%%%% LyX specific LaTeX commands.
\pdfpageheight\paperheight
\pdfpagewidth\paperwidth

%%%%%%%%%%%%%%%%%%%%%%%%%%%%%% User specified LaTeX commands.
\renewcommand{\appendixtocname}{Apendices}
\renewcommand{\appendixpagename}{Apendices}
\newcommand{\rnaemo}{\textbf{\emph{RNAemo}}}

\title{\textbf{RNAemo}\\ \vspace{0.45cm} Especificación de Requerimientos de Software} %ver cual es el paquete de los acentos
\author{Franco Gaspar Riberi}

\makeatother

\begin{document}
\maketitle\pagebreak{}\tableofcontents{}\pagebreak{}

\section*{Tabla de Revisiones}
	
\begin{center}
\begin{tabular}{| c | c | c | l |}
	\hline
	{\bf Fecha} & {\bf Autor} & {\bf Version} & {\bf Cambios}\\
	\hline
	\hline		
	12-03-12 & Franco Riberi & 0.1 & Template inicial.\\\hline
	13-03-12 & Franco Riberi & 0.2 & Sección 1.1, 1.2, 1.3,  y 1.5.\\\hline
	14-03-12 & Franco Riberi & 0.3 & Sección 1.6 y 1.7\\\hline
	15-03-12 & Franco Riberi & 0.4 & Sección 1.4, 2.1 \\\hline
	20-03-12 & Franco Riberi & 0.5 & Sección 2.2 y 2.3 \\\hline
	21-03-12 & Franco Riberi & 0.6 & Sección 2.4, 2.5 \\\hline
	26-03-12 & Franco Riberi & 0.7 & Sección 3.1 \\\hline
	27-03-12 & Franco Riberi & 0.7 & Sección 3.2, 3.3, 3.4, 3.5 y 4.1 \\\hline
	%3.1
\end{tabular}
\end{center}
\newpage

\section{Introducción}
En esta sección se describe un panorama completo del SRS.

\subsection{Propósito}
\par El propósito de este documento es la especificación de requerimientos
de software en el marco de la tesis de grado de la carrera Licenciatura en
Ciencias de la Computación de la UNRC, \textbf{``}\rnaemo\textbf{''}.  Los requerimientos 
del software son provistos por integrantes de FuDePAN en su carácter de autores
intelectuales de la solución que se pretende implementar y colaboradores
de dicha tesis.
\par Además, este documento establece la primera etapa de dicha tesis y será utilizado
como parte de la validación final del proyeto.

\subsection{Convenciones del Documento}
Las palabras clave DEBE, NO DEBE, REQUERIDO, DEBERÁ, NO DEBERÁ, DEBERÍA, NO DEBERÍA, RECOMENDADO, PUEDE y OPCIONAL
en este documento son interpretadas como esta descripto en el documento RFC
2119 [8]. 

\subsection{Audiencia Esperada}
\par A continuación se enumeran las personas involucradas en el desarrollo de la
tesis y que por lo tanto, representan la principal audiencia del presente documento.
\begin{itemize}
	\item Dr. Roberto Daniel Rabinovich: Miembro del INBIRS. Anteriormente CNRS, Centro Nacional de Referencia para el SIDA(Departamento Microbiología 											UBA). Profesor Titular de Virología (Departamento Ciencias Biológicas, CAECE). Colaborador de FuDePAN.
	\item Lic. Lucía Fazzi: Licenciada en Genética.
	\item Maria Pilar Adamo: Colaborador de tesis, FuDePAN. 
	\item Daniel Gutson: Colaborador de tesis, FuDePAN. 
	\item Lic. Guillermo Biset: Colaborador de tesis, FuDePAN. 
	\item Lic. Laura Tardivo: Directora de tesis, UNRC. 
	\item Ac. Franco Gaspar Riberi: Tesista, UNRC.
\end{itemize}

\subsection{Alcance}
\par El producto que se especifica en este documento se denomina ``\rnaemo''
 y su principal objetivo es contrastar formalmente una idea encomendada y
postulada por el Dr. Roberto Daniel Rabinovich que involucra principalmente 
la molécula de RNA. 
\par Dado un conjunto de small-RNA$_s$\footnote{Moléculas de RNA muy pequeñas, 
dentro de las cuales se encuentra la $_m$$_i$RNA, $_s$$_i$RNA, entre otras.} y
 una colección de RNA$_m$ de un determinado virus, dado que hay un RNA$_m$ por
 cada gen, determinar si existen small-RNA$_s$ que se van a pegar al RNA$_m$. 
Luego contabilizar la cantidad de small-RNA$_s$ que se pegan tanto a la secuencia
 de RNA$_m$ original como a la secuencia complementaria. De manera similar, recorrer 
el degradé de deshumanización total a humanización total\footnote{La deshumanización total
 y humanización total refiere a mutar de forma silente toda la secuencia. La diferencia radica
 en si los tripletes mutados son los preferenciales o no.} contabilizando la cantidad 
de small-RNA$_s$ que se pegan. Discriminar en cuanto a la célula.
\par Que se pegue o no un small-RNA$_s$ a un determinado RNA$_m$ involucra distintas
 reglas, siendo la más importante la complementariedad de bases. Pero además, existen 
otras tales como la estructura secundaria de la molécula.

\par Dadas estas dos reglas, la cuantificación de small-RNA$_s$ se determinará teniendo en cuenta:
\begin{enumerate}
	\item Complementariedad.
	\item Estructura secundaria.
\end{enumerate}

Determinar el impacto de la estructura secundaria puede dar respuesta a lo siguiente:
\begin{enumerate}
	\item Estabilidad del RNA$_m$: $\Delta$(G). Cantidad de bases apareadas o desapareadas.
	\item Unión entre $_m$$_i$RNA con RNA$_m$. ($_m$$_i$RNA / RNA$_m$).
	\item Estabilidad de ($_m$$_i$RNA / RNA$_m$). Es decir, cuan buena es la unión según la estructura secundaria.
\end{enumerate}

%En primera instancia, el proyecto involucrará el Virus de la Polio. 
\par El proyecto involucrará un determinado virus aún no definido. Para tal fin se aplicarán criterios biológicos.

\par El sistema a desarrollar comprenderá las siguientes característica:
\begin{itemize}
	\item Abarcar en su totalidad los requerimientos del problema.
	\item Construir un sistema que puede ser extendido en otros proyectos, brindando un diseño flexible. 
	\item Que proponga un buen uso de las prácticas de diseño para su mejor desempeño.
	\item Que posea abundante documentación clara y precisa.
	\item Lograr un código fuente bien escrito y estructurado respetando las buenas pácticas de programación.
\end{itemize}

\subsection{Descripción general del documento}
\par La estructura de este documento sigue las recomendaciones de la ``Guía para
la especificación de requerimientos de la IEEE'' (IEEE Std 830-1998) [1].
El documento esta organizado en las siguientes secciones generales: 
\par En la \textit{sección 1}, Introducción, se presenta una primera aproximación al proyecto. 
\par En la \textit{sección 2}, Descripción General, se presenta una descripción general de \rnaemo, sus principales
funcionalidades, interfaces y perfiles de usuarios. 
\par En la \textit{sección 3}, Requerimientos, se detallan los requerimientos funcionales específicos de \textbf{\emph{RNAemo}} y
 los principales atributos que debe cumplir el software.
\par Por último, en la \textit{sección 4} se detalla un pequeño apéndice.

\subsection{Referencias}
[1] IEEE Recommended Practice for Software Requirements Specifications. Copyright © 1998 by the Institute of Electrical and Electronics Engineers, Inc. All rights reserved. Published 1998. Printed in the United States of America. ISBN 0-7381-0332-2. \\

[2] SOLID: ``Design Principles and Design Patterns'', Robert C. Martin. \textcolor{blue}{http://www.objectmentor.com/resources/articles/Principles\_and\_Patterns.pdf} \\

[3] C++: Lenguaje de programación. \textcolor{blue}{http://www.cplusplus.com} \\

[4] G. Biset, D. Gutson, and M. Arroyo, “A framework for small distributed projects and a protein clusterer application”, 2009. \\

[5] G. Biset, D. Gutson, and M. Arroyo, “Fud: Design and implementation of a framework for small distributed applications”, 2009. \\

[6] B. Meyer, “Object-Oriented Software Construction”, Second Edition, Santa Barbara: Prentice Hall Professional Technical Reference, 1997. \\

[7] G. Booch, J. Rumbaugh, and I. Jacobson, “Unified Modeling Language User Guide”, Second Edition, 2005. \\

[8] RFC 2119. \textcolor{blue}{http://tools.ietf.org/html/rfc2119} \\

[9] GNU General Public License. \textcolor{blue}{http://www.gnu.org/licenses/} \\

[10] H. Curtis, N. Sue Barnes, A. Schnek and G. Flores, “Biología”, Editorial Médica Panamericana S.A, 2006, ISBN: 950-06-0423-X. \\

[11] B. Pierce, “Genética. Un enfoque conceptual”, Tercera Edición, Editorial médica panamericana S.A, ISBN: 978-84-9835-216-0. \\

[12] A. Blanco, “Química Biológica”, Séptima Edición, Editorial El Ateneo. \\

[13] $\Delta$(G): \textcolor{blue}{http://en.wikipedia.org/wiki/Gibbs\_free\_energy} \\

[14] FuD : \textcolor{blue}{http://code.google.com/p/fud/} \\

[15] fudepan-build: \textcolor{blue}{http://fudepan-build.googlecode.com} \\

[16] BLAST: \textcolor{blue}{http://blast.ncbi.nlm.nih.gov/Blast.cgi}

\section{Descripción General}
\label{section-desc-gral}
Esta sección describe los requisitos del producto de modo general. Los
requisitos específicos se describen en la sección 3.

\subsection{Perspectiva del Producto}
La bioinformática es una disciplina dedicada al análisis de elementos biológicos utilizando a la informática como herramienta principal para generar simulaciones, probar teorías, o realizar cálculos complejos entre otros aspectos. En particular, el producto a desarrollar apunta al cálculo complejo de ciertos datos, los cuales son de gran interés para la biología. El mismo, no será un componente de un sistema de mayor envergadura, sino que por el contrario, será totalmente autónomo e independiente. 

	\subsubsection{Interfaces del Sistema}
		El producto será capaz de correr al menos en sistemas GNU/Linux, por lo cual sólo se utilizarán librerías 			compatibles con el mismo.

	\subsubsection{Interfaces de Usuario}		
		En primera instancia, el usuario interactuará con el sistema mediante una CLI, no se proveerá de interface gráfica de usuario (GUI).

	\subsubsection{Interfaces de Hardware}
		El producto de software no requerirá hardware específico alguno para su correcto funcionamiento; cabe destacar 	que debido al uso de paralelismo, 			su performance mejorará en arquitecturas multi-core.

	\subsubsection{Interfaces de Software}
		Las librerías requeridas para el funcionamiento del sistema son las siguientes:
		\begin{enumerate}
			\item \textbf{BioPP:} Librería de C++ para el manejo de estructuras biológicas, código
						genético, entre otras funciones. 
						\par \noindent \textcolor{blue}{http://code.google.com/p/biopp/http://code.google.com/p/biopp/}

			\item \textbf{FuD:} Framework para la implementación de aplicaciones distribuidas. 
						\par \noindent \textcolor{blue}{http://code.google.com/p/fud/http://code.google.com/p/fud/}

			\item \textbf{MiLi:} Es una colección de pequeñas bibliotecas C++, compuesta unicamente por headers. Sin necesidad de instalación, sin un 									makefile, sin complicaciones. Soluciones simples para problemas sencillos.
						\par \noindent \textcolor{blue}{http://code.google.com/p/mili/http://code.google.com/p/mili/}

			\item \textbf{librnafold:} \textbf{\textcolor{red}{COMPLETE}}
		\end{enumerate}

	\subsubsection{Interfaces de Comunicaciones}
		No hay requerimientos especificados.

	\subsubsection{Restricciones de Memoria}	
		\par Este proyecto no presenta restricciones en cuanto a la cantidad de memoria mínima necesaria para operar. El sistema DEBERÁ manejar la memoria 			en forma correcta y sin que ocurran memory leaks.
		\par Las dependencias externas que serán utilizadas en el producto no deberán ser tenidas en cuenta en el chequeo 
		de memory leaks u otros problemas.
		
	\subsubsection{Operaciones}
		El modo de operación del sistema será: 
		\begin{enumerate}
			\item Invocación por consola por parte del usuario. Esta invocación DEBERÁ tener asociada los siguiente parámetros:
				\begin{itemize}
					%\item BD small-RNA$_s$
					\item Colección de RNA$m$ que codifique un determinado virus.
					\item Cantidad de random por RNA$_m$.
					\item Discretización del eje X				
					\item Argumento de \% de matching (min,max,step).
				\end{itemize}
			\item Realización de cálculos internos.
			\item Exhibición de resultados en caso de que no se produzcan situaciones erroneas, de lo contrario, el sistema DEBERÁ informar de tal suceso.
		\end{enumerate}

	\subsubsection{Requerimientos de Instalación}
		Como build system e instalador de código se usará fudepan-build [15].

\subsection{Funciones del Producto}
	El sistema DEBERÁ:
	\begin{itemize}
		\item Tomar como entrada diferentes parámetros, entre ellos una colección de RNA$_m$, la cantidad de random por RNA$_m$, la discretización del 				eje	X y un argumento de \% de matching.  
			% Colección de RNAm de un determinado virus, (dado que hay un RNAm por cada gen).
			% Cantidad de random por RNAm: por cada mensajero, cuantas secuencias vas a hacer random para hacer el control.
			%							   Hacer random se refiere a permutaciones al azar para conservar la misma cantidad de cada nucleótido
			%							   (y mismo tamaño).
			%							   EJ: A A T T C C G G -->	(1 random puede ser) A T A T G C G C			
			% Discretizacion del eje X: cantidad de puntos a tomar entre 0 y 1.
			%							0- todos los tripletes son los no humanizados. (puede haber varias secuencias con 0)
			%							1- todos los tripletes son los preferidos por el genoma humano.
			% Argumento de \% de matching:

		\item Generar una conexión con la base de datos de $_s$$_m$$_a$$_l$$_l$RNA$_s$.
		\item Generar la secuencia complementaria del RNA$_m$.
		\item Hallar secuencias similares según cierto porcentaje.
		\item Realizar cálculos internos.
		\item Exhibir los resultados.
			\begin{itemize}
				\item Exhibición de tablas comparativas de posibles interferencias por $_m$$_i$RNA para el virus presente en la naturaleza y el virus 					humanizado.
				\item Exhibición de un gráfico con resúmenes estadísticos.
			\end{itemize}

	\end{itemize}

\subsection{Características de Usuario}
	Se identifican tres tipos de usuarios para el sistema:
	\begin{enumerate}
 		\item \textit{Usuario final:} este tipo de usuario refiere a aquellas personas profesionales que utilizarán el 										producto. Sólo deberán interactuar gráficamente (mediante CLI), cargando los datos de entrada 										necesarios y ejecutando el programa para luego obtener el resultado. 
		\item \textit{Usuario de extensión:} refiere a profesionales con conocimientos biológicos capaces de utilizar los 												posibles resultados para nuevas investigaciones o trabajos.
		\item \textit{Usuario desarrollador:} refiere a usuarios con ciertos conocimientos específicos en programación, 											que serán los encargados de ampliar o extender las funcionalidades del 												sistema, como así también posibles mejoras algorítmicas.
	\end{enumerate}

\subsection{Restricciones}
	El producto DEBERÁ ser desarrollado utilizando el lenguaje de programación C++ [3] y bajo la licencia de software 		GPLv3 [9]. Además, se DEBERÁ respetar los lineamientos generales impuestos por FuDePAN (Thesis Guideline y Coding Guideline). 

\subsection{Trabajo Futuro}
A partir del resultado obtenido por el presente sistema, en futuras iteraciones se puede utilizar el mismo para definir moléculas de RNA interferentes. Esto involucra dos situaciones:
\begin{enumerate}
	\item Dada una cierta heurística aún no definida, encontrar un target a interferir (pequeña secuencia de nucleótidos), el cual será atacado por el RNA 			interferente.
	\item Encontrar la secuencia óptima para interferir ese target elegido. Se debe tener en cuenta, que el proceso de folding está relacionado 			al factor temperatura del huésped. Es decir, diseñar RNA que expresen secuencias sensibles al modificar el factor temperatura.
\end{enumerate}


\section{Requerimientos}
\label{section-req} 
En esta sección se detallan específicamente los requerimientos del producto. 

\subsection{Funciones del Sistema}

	\subsubsection{Interfaces Externas}
		No hay requerimientos especificados.
		
	\subsubsection{Requerimientos Funcionales}
	\begin{itemize}
		\item \textbf{Nombre del requerimiento:} Verificar collección de RNA$_m$. (RF1).\\
 	    \textbf{Propósito:} \\
		\textbf{Input:} \\
		\textbf{Procesamiento:} \\
		\textbf{Output:} \\

		\item \textbf{Nombre del requerimiento:} Verificar el número de random por cada RNA$_m$. (RF2).\\
 	    \textbf{Propósito:} \\
		\textbf{Input:} \\
		\textbf{Procesamiento:} \\
		\textbf{Output:} \\

		\item \textbf{Nombre del requerimiento:} Verificar el número discretización del eje X (RF3).\\
 	    \textbf{Propósito:} \\
		\textbf{Input:} \\
		\textbf{Procesamiento:} \\
		\textbf{Output:} \\

		\item \textbf{Nombre del requerimiento:} Verificar el argumento de \% de matchig. (RF4).\\
 	    \textbf{Propósito:} \\
		\textbf{Input:} \\
		\textbf{Procesamiento:} \\
		\textbf{Output:} \\

		\item \textbf{Nombre del requerimiento:} Informar errores en la entrada. (RF5).\\
 	    \textbf{Propósito:} \\
		\textbf{Input:} \\
		\textbf{Procesamiento:} \\
		\textbf{Output:} \\

		\item \textbf{Nombre del requerimiento:} Obtener la secuencia complementaria de una secuencia original dada.\\
		\textbf{Propósito:} dada una secuencia, llamamemosla cadena original, obtener su cadena complementaria a partir del principio de 										complementariedad de bases (A=T,G=C)\\
		\textbf{Input:} secuencia original\\
		\textbf{Procesamiento:} \\
		\textbf{Output:} secuencia complementaria\\

		\item \textbf{Nombre del requerimiento:} Generar secuencia humanizada de RNA$_m$. (RF).\\
  	    \textbf{Propósito:} \\
		\textbf{Input:} \\
		\textbf{Procesamiento:} \\
		\textbf{Output:} \\


		\item \textbf{Nombre del requerimiento:} Permutar al azar una secuencia de RNA$_m$. (RF).\\
  	    \textbf{Propósito:} \\
		\textbf{Input:} \\
		\textbf{Procesamiento:} \\
		\textbf{Output:} \\


		\item \textbf{Nombre del requerimiento:} Determinar si un $_s$$_m$$_a$$_l$$_l$RNA$_s$ se hibrida a una secuencia. (RF).\\
  	    \textbf{Propósito:} \\
		\textbf{Input:} \\
		\textbf{Procesamiento:} \\
		\textbf{Output:} \\
					
		\item \textbf{Nombre del requerimiento:} Hallar secuencia de matching. (RF).\\
  	    \textbf{Propósito:} \\
		\textbf{Input:} \% de matching, mix,max y step.\\
		\textbf{Procesamiento:} \\
		\textbf{Output:} \\

		\item \textbf{Nombre del requerimiento:} Realizar cálculos internos (RF).\\
  	    \textbf{Propósito:} \\
		\textbf{Input:} \\
		\textbf{Procesamiento:} \\
		\textbf{Output:} \\


		\item \textbf{Nombre del requerimiento:} Exhibir tabla comparativas (RF).\\
  	    \textbf{Propósito:} \\
		\textbf{Input:} \\
		\textbf{Procesamiento:} \\
		\textbf{Output:} \\

		\item \textbf{Nombre del requerimiento:} Exhibir gráfico resultado (RF).\\
  	    \textbf{Propósito:} \\
		\textbf{Input:} \\
		\textbf{Procesamiento:} \\
		\textbf{Output:} \\

	\end{itemize}


%requerimientos no funcionales
\subsection{Restricciones de Rendimiento}
No hay requerimientos especificados.

\subsection{Base de Datos}
	\par El sistema requerirá de una base de datos de small-RNA$_s$. En caso alternativo, se permitirá el uso de BLAST para generar secuencias. 

\subsection{Restricciones de Diseño}
\par El producto DEBERÁ cumplir con los siguientes principios de diseño de la
programación orientada a objetos. Los 5 primeros, son también conocidos por
el acrónimo \textbf{``SOLID''} [2].
\begin{itemize}
	\item \textbf{S}ingle responsibility principle (SRP)
	\item \textbf{O}pen/closed principle (OCP)
	\item \textbf{L}iskov substitution principle (LSP)
	\item \textbf{I}nterface segregation principle (ISP)
	\item \textbf{D}ependency inversion principle (DIP)	
	\item Law of Demeter (LoD)
\end{itemize}

\subsection{Atributos del Software}
\par El código del producto DEBERÁ:
\begin{itemize}
 \item Compilar sin advertencias, o las advertencias aceptadas DEBERÁN estar documentadas.
 \item Cumplir con el estándar ANSI C++ y el ``\textit{coding style}'' definido por FuDePAN.
\end{itemize}
\par El software DEBERÁ:
\begin{itemize}
	\item Funcionar sin memory leaks.
	\item Tener al menos un 85\% de cobertura con pruebas automatizadas.
\end{itemize}

\section{Apéndice}
\subsection{Definiciones, Acrónicos y Abreviaturas}
\begin{itemize}
	\item \textbf{RNAemo:} Nombre que recibe el presente producto.
	\item \textbf{UNRC:} Universidad Nacional de Río Cuarto.
	\item \textbf{FuDePAN:} Fundación para el Desarrollo de la Programación en Ácidos Nucleicos.
	\item \textbf{INBIRS:} Instituto Biomédico en Retrovirus y SIDA.
	\item \textbf{SIDA:} acrónimo de síndrome de inmunodeficiencia adquirida. También abreviada como VIH-sida o VIH/sida.
	\item \textbf{CAECE:} Centro de Altos Estudios en Ciencias Exactas.
	\item \textbf{IEEE:} Institute of Electrical and Electronics Engineers.
	\item \textbf{SOLID:} acrónimo nemotécnico introducido por Robert C. Martin en la
							década de 2000, que representa cinco principios básicos de programación
							y diseño orientado a objetos
	\item \textbf{GPL:} \textit{G}eneral \textit{P}ublic \textit{L}icense.	
	\item \textbf{SRS:} Especificación de requerimientos.
	\item \textbf{FuD:} FuDePAN Ubiquitous Distribution [14]. Framework para el desarrollo de aplicaciones distribuídas a través de disposiciones 							heterogéneas y dinámicas de nodos de procesamientos.
	\item \textbf{CLI:} Interfaz de Línea de Comandos, por su acrónimo en inglés de Command Line Interface. Permite dar instrucciones a algún programa 							informático por medio de una línea de texto.
	\item \textbf{Nucleótido:} molécula orgánica formada por la unión covalente de un monosacárido de cinco carbonos (pentosa), una base nitrogenada y un 								   grupo fosfato.
	\item \textbf{Triplete:} es un grupo de nucleótidos que corresponden a un aminoácido concreto, también conocido como codón.
	\item \textbf{Aminoácido:} molécula orgánica que conforma la proteína.
	\item \textbf{RNA:} Ácido ribonucléico. Es un tipo de ácido nucleico compuesta por nucléotidos esencial para la vida.
	\item \textbf{DNA:} Ácido desoxirribonucleíco. Es un tipo de ácido nucleico, forma parte de todas las células.
	\item \textbf{RNA$_m$:} RNA mensajero. Se encuentra tanto en el núcleo como en el citoplasma celular. Su función es portar el código genético para
							las proteínas, es decir, transportan las instrucciones de codificación de las proteínas desde el DNA.
	\item \textbf{$_s$$_i$RNA:} short interfering RNA. Corresponde a una clase de RNA de cadena doble presente en células eucariotas.
	\item \textbf{$_m$$_i$RNA:} micro RNA. Corresponde a una clase de RNA de cadena simple presente en células eucariotas. 
	\item \textbf{small-RNA$_s$:} Moléculas muy pequeñas de RNA. Dentro de la clasificación de RNA, aparecen como RNA no codificante.
	\item \textbf{Proteína:} macromolécula formada por cadenas lineales de aminoácidos. Se considera proteína a aquellas cadenas de aminoácidos  enlazados 								cuyo peso molecular es superior 6000 Daltons.
	\item \textbf{Virus:} Entidad biológica que para reproducirse necesita de una célula huésped.	
%	\item \textbf{Virus de la Polio:} virus que sólo infecta a los humanos. Afecta el sistema nervioso central, más específicamente, la sustancia gris en 										la médula y tronco del encéfalo.  
	\item \textbf{$\Delta$(G):} refiere a la energía libre. Es el potencial químico que se minimiza cuando un sistema alcanza el equilibrio a presión y 								temperatura constante. [13] 
%	\item \textbf{Proceso de melitación:}	
%	\item \textbf{Silenciamiento genético transcripcional:}	
%	\item \textbf{RNA codificante:} es aquel RNA que genera proteínas. Se encuentra el RNA$_m$.
	\item \textbf{RNA no codificante:} es aquel RNA que no genera proteínas. Se encuentra el RNA transcripcional y small-RNA$_s$.
	\item \textbf{Humanización-Deshumanización:} refiere a mutar de forma silente una secuencia. Esto significa, mutar nucleótidos de un triplete 													conservando la expresión del aminoácido. La diferencia entre humanización y deshumanización radica en que 													si los tripletes por los que se muta son o no los preferenciales.				
	\item \textbf{Mutación silente:} Las mutaciones silentes ocurren cuando se produce un cambio de un sólo nucleótido de DNA dentro de una porción de un 										gen codificador para una proteína que no afecta la secuencia de aminoácidos que componen la proteína para el gen. Un 										cambio en un nucleótido, sin embargo, no siempre cambia el significado de un triplete. El triplete mutado puede aún 									representar el mismo aminoácido. Y cuando los aminoácidos de una proteína siguen siendo los mismos, esta mantiene su 										estructura y función.				
	\item \textbf{BLAST:} \textit{B}asic \textit{L}ocal \textit{A}lignment \textit{S}earch \textit{T}ool [16]. Es un programa de alineamiento de secuencias, ya se de DNA, RNA o proteínas. Es capaz de comparar una secuencia problema (denominada query) contra una gran cantidad de secuencias almacenadas en una base de datos. Encuentra las secuencias de la base de datos que tienen mayor parecido a la secuencia query. BLAST es desarrollado por los Institutos Nacionales de Salud del gobierno de Estados Unidos.
\end{itemize}

\subsection{Manejo de inputs}
\par Para la manipulación de los datos se usaran cadenas de caracteres que representan tanto cadenas de DNA como cadenas de RNA para representar genes como nucleóticos.
\begin{itemize}
	\item \textbf{nuc\_arn} $\to$  a|u|c|g|\_ 
	\item \textbf{gen\_arn} $\to$ (\textbf{nuc\_arn})\textsuperscript{+}

	\item {nuc\_adn} $\to$ a|t|c|g|\_
	\item {gen\_adn} $\to$ (\textbf{nuc\_adn})\textsuperscript{+}
\end{itemize}
\par Para formar cadenas más complejas, tales como aminoácido y proteínas, se usará:
\begin{itemize}
 	\item \textbf{aminoacido} $\to$ Ala|Arg|Asn|...	
	\item \textbf{proteina} $\to$ \textbf{aminoacido}(\textbf{aminoacido})\textsuperscript{+}
\end{itemize}

\end{document}
