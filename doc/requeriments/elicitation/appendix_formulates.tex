\section{Sugerencias}
\label{appendix-formulates}

	\large \textbf{C.1 Pseudo-Código para enmascarar nucleótidos (``M'')}
	\par Para determinar que nucleótidos deben ser reemplazados por una ``M'' en la generación de secuencias enmascaradas, 		se propone el siguiente Pseudo-Código.

    \begin{verbatim}
        input: nuc_mensajero, nuc_mirna

        if (nuc_mensajero == complemento(nuc_mirna)) {
            if (nuc_mensajero.apareado()){
                print "M"
            }else{
                print upper_case(nuc_mirna)			
            }	
        }else{
                print lower (nuc_mirna)
        }                
    \end{verbatim}

	\large \textbf{C.2 Scores matching}
	\par Para calcular el score de matching sobre las secuencias (secuencia original, secuencia humanizada) se sugiere la 		fórmula (\ref{eq}).

	\begin{equation}	
		\label{eq} \frac{(\#AT \times constAT + \#GC \times constGC)}{(totalAT \times constAT + totalGC \times constGC}
	\end{equation}	

	\par donde:
		\begin{itemize}
			\item \textbf{\#AT:} cantidad de Adenina que hace matching con Timina, o viceversa.
			\item \textbf{\#GC:} cantidad de Guanina que hace matching con Citosina, o viceversa.
			\item \textbf{constAT:} valor predeterminado para el apareo A=T.
			\item \textbf{constGC:} análogo al anterior, pero con apareo G=C.
			\item \textbf{total AT:} total de adedina y timina (apareadas o no).
			\item \textbf{totalGC:} total de guanina y citosina (apareadas o no).	
		\end{itemize} 

	\par Esta fórmula permitirá calcular dos score, uno de ellos empleando constantes \textbf{constAT} y \textbf{constGC} 		de valor 1 (cuyo resultado corresponderá a un porcentaje), y para el otro se emplearan constantes de folding.
 
