\section{Sugerencias}
\label{appendix-formulates}

	\large \textbf{C.1 Pseudo-Código para enmascarar nucleótidos (``M'')}
	\par Para determinar que nucleótidos deben ser reemplazados por una ``M'' en la generación de secuencias enmascaradas, 		se propone el siguiente Pseudo-Código.

    \begin{verbatim}
        input: nuc_mensajero, nuc_mirna

        if (nuc_mensajero == complemento(nuc_mirna)) {
            if (nuc_mensajero.apareado()){
                print "M"
            }else{
                print upper_case(nuc_mirna)			
            }	
        }else{
                print lower_case(nuc_mirna)
        }                
    \end{verbatim}

	\large \textbf{C.2 Scores matching}
	\par Para calcular el score de matching sobre las secuencias (secuencia original, secuencia humanizada) se sugiere la fórmula (1).

	\begin{equation}	
		\label{eq} \emph{constAT} \times (\#MsgMicro\_a\_u - \#MsgMsg\_a\_u) \vskip{} \hspace*{6cm} + \vskip{} \hspace*{1.15cm} \emph{constGC} \times (\#MsgMicro\_c\_g - \#MsgMsg\_c\_g) 
	\end{equation}	

	\par donde:
		\begin{itemize}			
			\item \textbf{constAT:} valor predeterminado para el apareo A=T.
			\item \textbf{constGC:} análogo al anterior, pero con apareo G=C.
			\item \textbf{MsgMicro$_x$$_y$:} cantidad de complementos entre RNA$_m$ y $_m$$_i$RNA de tipo x-y. 
			\item \textbf{MsgMsg$_x$$_y$:}  cantidad de apareamientos en el folding del mensajero de tipo x-y. 
		\end{itemize} 

	\par Esta fórmula permitirá calcular dos score, uno de ellos empleando constantes \textbf{constAT} y \textbf{constGC} de valor 1 (cuyo resultado   
     corresponderá a un porcentaje), y para el otro se emplearan constantes de folding a determinar en el análisis.

    \par Observación: si en ambos términos de la fórmula, la cantidad de complementos micro-msg es menor a la cantidad de apareados msg-msg: 
    \begin{center} \textbf{-k $\times$ (x - y)} con x $<$ y. \end{center}
    \par Entonces el resultado será un número positivo. 
