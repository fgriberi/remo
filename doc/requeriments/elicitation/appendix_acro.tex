\section{Definiciones, Acrónicos y Abreviaturas}
\label{appendix-acro}
\begin{itemize}
	\item \textbf{RNAemo:} Nombre que recibe el presente producto.
	\item \textbf{UNRC:} Universidad Nacional de Río Cuarto.
	\item \textbf{FuDePAN:} Fundación para el Desarrollo de la Programación en Ácidos Nucleicos [17].
	\item \textbf{INBIRS:} Instituto Biomédico en Retrovirus y SIDA.
	\item \textbf{CNRS:} Centro Nacional de Referencia para el SIDA.
	\item \textbf{SIDA:} acrónimo de síndrome de inmunodeficiencia adquirida. También abreviada como VIH-sida o VIH/sida.
	\item \textbf{CAECE:} Centro de Altos Estudios en Ciencias Exactas.
	\item \textbf{IEEE:} Institute of Electrical and Electronics Engineers.
	\item \textbf{SOLID:} acrónimo nemotécnico introducido por Robert C. Martin en la
							década de 2000, que representa cinco principios básicos de programación
							y diseño orientado a objetos
	\item \textbf{GPL:} \textit{G}eneral \textit{P}ublic \textit{L}icense.	
	\item \textbf{SRS:} Especificación de requerimientos.
	\item \textbf{FuD:} FuDePAN Ubiquitous Distribution [14]. Framework para el desarrollo de aplicaciones distribuídas a través de disposiciones 							heterogéneas y dinámicas de nodos de procesamientos.
	\item \textbf{CLI:} Interfaz de Línea de Comandos, por su acrónimo en inglés de Command Line Interface. Permite dar instrucciones a algún programa 							informático por medio de una línea de texto.
	\item \textbf{Nucleótido:} molécula orgánica formada por la unión covalente de un monosacárido de cinco carbonos (pentosa), una base nitrogenada y un 								   grupo fosfato.
	\item \textbf{Tripletes:} conjunto de tres nucleótidos que determinan un aminoácido concreto, también conocido como codón.

	\item \textbf{Aminoácido:} molécula orgánica que conforma la proteína.
	\item \textbf{RNA:} Ácido ribonucléico. Es un tipo de ácido nucleico compuesta por nucléotidos esencial para la vida.
	\item \textbf{DNA:} Ácido desoxirribonucleíco. Es un tipo de ácido nucleico, forma parte de todas las células.
	\item \textbf{RNA$_m$:} RNA mensajero. Se encuentra tanto en el núcleo como en el citoplasma celular. Su función es portar el código genético para
							las proteínas, es decir, transportan las instrucciones de codificación de las proteínas desde el DNA.
	\item \textbf{$_s$$_i$RNA:} short interfering RNA. Corresponde a una clase de RNA de cadena doble presente en células eucariotas.
	\item \textbf{$_m$$_i$RNA o microRNA:} Corresponde a una clase de RNA de cadena simple presente en células eucariotas. 
	\item \textbf{small-RNA$_s$:} Moléculas muy pequeñas de RNA. Dentro de la clasificación de RNA, aparecen como RNA no codificante.
	\item \textbf{Proteína:} macromolécula formada por cadenas lineales de aminoácidos. Se considera proteína a aquellas cadenas de aminoácidos  enlazados 								cuyo peso molecular es superior 6000 Daltons.
	\item \textbf{Virus:} Entidad biológica que para reproducirse necesita de una célula huésped.	
%	\item \textbf{Virus de la Polio:} virus que sólo infecta a los humanos. Afecta el sistema nervioso central, más específicamente, la sustancia gris en 										la médula y tronco del encéfalo.  
	\item \textbf{$\Delta$(G) o energía libre:} Es el potencial químico que se minimiza cuando un sistema alcanza el equilibrio a presión y 								temperatura constante. [13] 
%	\item \textbf{Proceso de melitación:}	
%	\item \textbf{Silenciamiento genético transcripcional:}	
%	\item \textbf{RNA codificante:} es aquel RNA que genera proteínas. Se encuentra el RNA$_m$.
	\item \textbf{RNA no codificante:} es aquel RNA que no genera proteínas. Se encuentra el RNA transcripcional y small-RNA$_s$.
	\item \textbf{Humanización-Deshumanización:} refiere a mutar de forma silente una secuencia. Esto significa, mutar nucleótidos de un triplete 													conservando la expresión del aminoácido. La diferencia entre humanización y deshumanización radica en que 													si los tripletes por los que se muta son o no los preferenciales.				
	\item \textbf{Mutación silente:} Las mutaciones silentes ocurren cuando se produce un cambio de un sólo nucleótido de DNA dentro de una porción de un 										gen codificador para una proteína que no afecta la secuencia de aminoácidos que componen la proteína para el gen. Un 										cambio en un nucleótido, sin embargo, no siempre cambia el significado de un triplete. El triplete mutado puede aún 									representar el mismo aminoácido. Y cuando los aminoácidos de una proteína siguen siendo los mismos, esta mantiene su 										estructura y función.				
	\item \textbf{BLAST:} \textit{B}asic \textit{L}ocal \textit{A}lignment \textit{S}earch \textit{T}ool [16]. Es un programa de alineamiento de 										secuencias, ya se de DNA, RNA o proteínas. Es capaz de comparar una secuencia problema (denominada query) contra una 										gran cantidad de secuencias almacenadas en una base de datos. Encuentra las secuencias de la base de datos que tienen 										mayor parecido a la secuencia query. BLAST es desarrollado por los Institutos Nacionales de Salud del gobierno de 										Estados Unidos.
	\item \textbf{Codones sinónimos}: término más conocido como \textit{``codon usage bias''}. \textcolor{red}{COMPLETE}


	\item \textbf{Distancia de hamming:} se denomina distancia de Hamming a la efectividad de los códigos de bloque y depende de la diferencia entre una 											palabra de código válida y otra. Cuanto mayor sea esta diferencia, menor es la posibilidad de que un código válido 											se transforme en otro código válido por una serie de errores. A esta diferencia se le llama distancia de Hamming, y 										se define como el número de bits que tienen que cambiarse para transformar una palabra de código válida en otra 										palabra de código válida.
\end{itemize}
