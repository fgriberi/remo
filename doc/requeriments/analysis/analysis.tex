\documentclass[12pt,a4paper,spanish]{article}

\usepackage[utf8]{inputenc}
\pagestyle{headings}
\usepackage{babel}
\usepackage{xcolor}
\usepackage{amsmath}
\usepackage{amsfonts}
\usepackage{amssymb}
\usepackage[pdftex]{graphicx}
\usepackage[unicode=true] {hyperref}
\usepackage[toc, page, header]{appendix}

\makeatletter

%%%%%%%%%%%%%%%%%%%%%%%%%%%%%% LyX specific LaTeX commands.
\pdfpageheight\paperheight
\pdfpagewidth\paperwidth

\title{\textbf{RNAemo}\\ \vspace{0.45cm} Análisis de Requerimientos de Software} %ver cual es el paquete de los acentos
\author{Franco Gaspar Riberi}

\begin{document}
\maketitle\pagebreak{}\tableofcontents{}\pagebreak{}

\newpage


\section{Introducción}
	\par El presente documento corresponde al análisis de requerimientos correspondietes a
	la iteración número 1 de la tesis de grado \textit{\textbf{``RNAemo''}}.
	\par A continuación se detalla dicho análisis.

\section{Problema}
	\par Actualmente no hay evidencia respecto a si existe diferencia en cuanto a reconocimiento de miRNA en los RNAm virales y sus supuestos humanizados. 		Por lo cual es necesario determinar la capacidad de hibridación de los microRNA en la secuencia viral tal como existe en la naturaleza y en la 	
	secuencia humanizada para poder establecer conclusiones al respecto. En otras palabras, determinar si el uso de codones divergente con respecto al 		huesped podría ser una consecuencia de la presión evolutiva generada por los miRNA. Si eso es así, los miRNA deberían tener menor capacidad de unirse 		al RNA viral que al genoma viral humanizado. 

	\par En principio esa respuesta es importante desde el punto de vista biológico, y daría una herramienta importante para desarrollos posteriores,
	por ejemplo actualmente se está estudiando el uso de virus modificados para combatir canceres, el programa podría predecir que virus sería menos 		afectado por los miRMA en celulas cancerosas y tenerlo en cuenta en el diseño del virus modificado.

	\subsection{Solución propuesta}
		\par Para determinar la divergencia en la hibridación de los miRNA en la secuencia viral original y humanizada, se contabilizará por separado la 			cantidad de miRNA que se hibridan a una secuencia viral y a la humanizada. Con esta información, se calcularán dos score de matching por cada 			genoma de las secuencias (viral y humanizada). El primero, dará un porcentaje de hibridación; el segundo, involucrará valores para las uniones 			\textsc{A=T} y \textsc{G=C} que permitirán calcular un score estimando burdamente la energía libre de estas uniones.

\section{Algoritmos y Métodos}
	\par Para calcular el score de matching sobre las secuencias se utilizará la siguiente fórmula:

		\begin{equation}	
			\frac{(\#AT \times constAT + \#GC \times constGC)}{(totalAT \times constAT + totalGC \times constGC}
		\end{equation}	

		\par donde:
		\begin{itemize}
			\item \textbf{\#AT:} cantidad de Adenina que hace matching con Timina, o viceversa.
			\item \textbf{\#GC:} cantidad de Guanina que hace matching con Citosina, o viceversa.
			\item \textbf{constAT:} valor predeterminado para el apareo A=T.
			\item \textbf{constGC:} análogo al anterior, pero con apareo G=C.
			\item \textbf{total AT:} total de adedina y timina (apareadas o no).
			\item \textbf{totalGC:} total de guanina y citosina (apareadas o no).	
		\end{itemize}

	\par \textbf{constAT} y \textbf{constGC} tendrán diferentes valores según el score que se calculé. Si se calcula el score de matching en porcentaje, las constantes tendrán valor 1, pero en contrapartida, si se calcula el score de matchig teniendo en cuenta las uniones, se utilizarán los siguiente valores:
	\begin{itemize}
		\item X para \textbf{constAT}. \textcolor{red}{COMPLETE WITH DANIEL G constantes de Zuker}
		\item Y para \textbf{constGC}. \textcolor{red}{COMPLETE WITH DANIEL G constantes de Zuker}
	\end{itemize}

\section{Búsqueda de datos mínimos}
	\par Esta etapa consistió en la relocección de datos necesarios para posteriormente realizar las primeras pruebas del producto a desarrollar. Esta 		fase involucró la recolección de RNA mensajeros, miRNA, constantes para calcular el score de matching teniendo las uniones (A=T, G=C) y software 		humanizador de secuencias.

	\subsection{RNAm}
		\par La bases de datos biológicas son DNA céntricas, es decir que si un virus tiene su información como RNA como la mayoría de los virus humanos, 			la secuencia archivada esta expresada como DNA (\textbf{U}racilo en lugar de \textbf{T}imina). Por lo cual, no se encontró RNAm directamente, sino 			que como gen.
		%%%%%%%%%%%%%%%%%%%%%%%%%%%%%%%%%%%%%%%%%%%%%%%%%%%%%%%%%%%%%%%%%%%%%%%%%%%%%%%%%%%%%%%%%%%%%%%%%%%%%%%%%%%%%%%%%%%%%%%%%%%%%%%%
		\par La búsqueda se basó en la nucleoproteina del virus del sarampión. Como 
		resultado\textcolor{blue}{http://www.ncbi.nlm.nih.gov/nuccore/FJ226465} se obtvó una secuencia de DNA en formato FASTA.
		\textcolor{blue}{COMPLETE}

		%%%%%%%%%%%%%%%%%%%%%%%%%%%%%%%%%%%%%%%%%%%%%%%%%%%%%%%%%%%%%%%%%%%%%%%%%%%%%%%%%%%%%%%%%%%%%%%%%%%%%%%%%%%%%%%%%%%%%%%%%%%%%%%%
		%Google -> Entrez Pubmed NCBI
		%ventanita -> measles nucleoprotein 
		%opciones porque hay varias secuencias, que son todas muy parecidas elegi por ejemplo la de Bangkock (Bangkok.THA/5.01)

	\subsection{miRNA}
		\par Se obtuvo una base de datos de miRNA en formato FASTA. Dicha base de datos fue descargada de \textcolor{blue}{COMPLETE}. De la misma, se 			selecionaron al azar apróximadamente 50 miRNA para luego realizar las primeras pruebas. Los miRNA estan formados por 20 nucleótidos 		
		aproximadamente. A continuación se exhiben algunas de las secuencias: 
			\par \hspace*{0.75cm} • \textsc{...}
			\par \hspace*{0.75cm} • \textsc{...}
			\par \hspace*{0.75cm} • \textsc{...}
			\par \hspace*{0.75cm} • \textsc{...}
			\par \hspace*{0.75cm} • \textsc{...}
			\par \hspace*{0.75cm} • \textsc{...}
		\par Otros posibles lugares de descarga de base de datos de miRNA consultados fueron: \textcolor{blue}{http://micrornadatabase.com/},
		\textcolor{blue}{http://microrna.org/}, \textcolor{blue}{http://mirdb.org/miRDB/}. \textcolor{blue}{http://www.mirbase.org/}

	\subsection{Determinación de constantes para calcular score}
		\textcolor{red}{Están en la página de Zuker. DANIEL G}

\section{Control sobre las secuencias}	
	\par Es necesario que las secuencias se encuentrn en el marco de lectura correcto, de lo contrario segnifica que se perdió un nucleótido y todos los 		tripletes están cambiados. Para comprobar esto, es necesario pasar la cadena de nucleótidos a una cadena de aminoácidos. Esta tarea se realizará
	utilizado la herramienta \textsc{Bioedit}, como así también se haŕa uso de la librería \textsc{BioPP} la cual provee esta funcionalidad.	
	Al realizar el cambio, no deben aparecer codones stop (representados por un asterisco). Si no aparecen codones stop quiere decir que las secuencias 	están en el marco correcto de lectura (in frame).
	%BioEdit Ctrl+G -> lo pasa temporalmente a aminoácidos
	
\section{Herramientas} 
	A continuación se describen las herramientas o software necesarios para el desarrollo del producto.
	\subsection{R}
		\par \textsc{R}\footnote{\textcolor{blue}{http://www.r-project.org}} es un entorno especialmente diseñado para el tratamiento de datos, cálculo y 			desarrollo gráfico. Permite trabajar con facilidad con vectores y matrices y ofrece diversas herramientas para el análisis de datos.
		\par El lenguaje de programación \textsc{R} forma parte del proyecto GNU\footnote{El proyecto GNU se inició en 1984 con el propósito de 	
		desarrollar un sistema operativo compatible con Unix que fuera de software libre. \textcolor{blue}{http://www.gnu.org/}} y puede verse como una 		implementación alternativa del lenguaje \textsc{S}, desarrollado en AT\&T Bell Laboratories. Se distribuye bajo la licencia GNU GPL y está 			disponible en una gran variedad de sistemas UNIX, como asi también en Windows y MacOS.
		\par \textsc{R} también es un entorno en el que se han ido implementando diversas técnicas estadísticas. Algunas de ellas se encuentran en la base 			de R pero muchas otras están disponibles como paquetes (packages\footnote{Estos paquetes están disponibles en 
		\textcolor{blue}{http://cran.au.r-project.org/.}}).
		\par En resumen, R proporciona un entorno de trabajo especialmente preparado para el análisis estadístico de datos. Sus principales 		
		características son:
		\begin{itemize}
			\item Proporciona un lenguaje de programación propio. Basado en el lenguaje S, que a su vez tiene muchos elementos del lenguaje C. 
			\item Objetos y funciones específicas para el tratamiento de datos.
			\item Es software libre. Permite la descarga de librerías con implementaciones concretas de funciones gráficas, métodos estadísticos, etcétera.
		\end{itemize}
		\par \textsc{R} será utilizado como software para graficar. Se utilizarán los scripts generados por SAARNA\footnote{\textcolor{blue}		
		{saarna.googlecode.com}}

	\subsection{Software para Humanizar}
		- OPTIMIZER, JCAT, y uno en la página GeneDesign, \textcolor{blue}{oGeneDesign}. \textcolor{red}{DANIEL R} 
	
		%https://www.dna20.com/secure/order.php?page=genedesigner2
		%/opt/DNA2.0 ubicacion

	\subsection{BioEdit}		% http://www.mbio.ncsu.edu/BioEdit/bioedit.html --> diversas versiones
								%:C/BioEdit wine
		\par \textsc{BioEdit} es un editor de alineación de secuencias biológicas escrito para Windows 95/98/NT/2000/XP/7. Permite la alineación y manipulación de secuencias relativamente fácil.

	\subsection{Jalview}
		\par \textsc{Jalview}\footnote{http://www.jalview.org/} es un editor de múltiples alineación escrito en Java, multiplataforma. El programa fue 			originalmente escrito por Michele Clamp\footnote{\textcolor{red}{complete}}. Se utiliza ampliamente por una variedad de servidores web (e.g., el 			servidor EBI ClustalW y la base de datos de proteínas Pfam dominio), pero está disponible como un editor de efectos de alineación general. 			\textsc{Jalview} tiene una amplia gama de funciones además de la alineación de secuencias múltiples, permite la visualización y edición incluyendo 			el cálculo de los árboles filogenéticos y la visualización de estructuras moleculares. 
		%http://www.jalview.org/Web\_Installers/install.htm

		%Linux Instructions:
		%Instructions

		%After downloading open a shell and,  cd to the directory where you downloaded the installer.
		%At the prompt type:  sh ./install.bin

		%Notes

		%If you do not have a Java virtual machine installed, be sure to download the package above which includes one. Otherwise you may need to download 			one from Sun's Java web site or contact your OS manufacturer.

		%(Go To Top)

	\subsection{FASTA}
		\par \textsc{FASTA} \cite{1} es un formato de archivo informático basado en texto, utilizado para representar secuencias de ácidos nucleicos, o de 			péptido, y en el que los pares de bases o los aminoácidos se representan usando códigos de una única letra. El formato también permite incluir 			nombres de secuencias y comentarios que preceden a las secuencias en sí.	

		\par Una secuencia bajo formato \textsc{FASTA} comienza con una descripción en una única línea (línea de cabecera), seguida por líneas de datos de 			secuencia. La línea de descripción se distingue de los datos de secuencia por un símbolo ``$>$''. La palabra siguiente a este símbolo es el 		identificador de la secuencia, y el resto de la línea es la descripción (ambos son opcionales). No debería existir espacio entre el ``$>$'' y la 			primera letra del identificador. La secuencia termina si aparece otra línea comenzando con el símbolo ``$>$'', lo cual indica el comienzo de otra 			secuencia. 
		\par A continuación se exhibe un ejemplo de una secuencia en tal formato:	
		\begin{verbatim}
				>gi|5524211|gb|AAD44166.1| cytochrome b [Elephas maximus maximus]
				LCLYTHIGRNIYYGSYLYSETWNTGIMLLLITMATAFMGYVLPWGQMSFWGATVITNLFSAIPYIGTNLV
				EWIWGGFSVDKATLNRFFAFHFILPFTMVALAGVHLTFLHETGSNNPLGLTSDSDKIPFHPYYTIKDFLG
				LLILILLLLLLALLSPDMLGDPDNHMPADPLNTPLHIKPEWYFLFAYAILRSVPNKLGGVLALFLSIVIL
				GLMPFLHTSKHRSMMLRPLSQALFWTLTMDLLTLTWIGSQPVEYPYTIIGQMASILYFSIILAFLPIAGX
				IENY
		\end{verbatim}

		\subsubsection{Representación de secuencias}	
			\par Cada línea de una secuencia debería tener algo menos de 80 caracteres. Las secuencia pueden corresponder a secuencias de proteínas 			 o de ácidos nucleicos, y pueden contener huecos (o gaps) o caracteres de alineamiento.
			 \begin{itemize}
				\item Los códigos de ácidos nucléicos soportados son:
					\begin{center}
						\begin{tabular}{| c | c |}
							\hline
							{\bf Código ácido nucléico} & {\bf Significado} \\
							\hline
							\hline		
							- &	hueco (gap) de longitud indeterminada \\\hline
							A & Adenosina \\\hline
							B &	G T C (no A) (B viene tras la A)	\\\hline 
							C &	Citosina \\\hline
							D &	G A T (no C) (D viene tras la C) \\\hline
							G &	Guanina \\\hline
							H &	A C T (no G) (H viene tras la G) \\\hline
							K &	G T (cetona/Ketone) \\\hline
							M &	A C (grupo aMino) \\\hline
							N &	A G C T (cualquiera/aNy) \\\hline
							R &	G A (puRina) \\\hline
							S &	G C (interacción fuerte/Strong interaction) \\\hline
							T &	Timidina \\\hline
							U &	Uracilo \\\hline
							V &	G C A (no T, no U) (V viene tras la U) \\\hline
							W &	A T (interacción débil/Weak interaction) \\\hline
							X &	máscara \\\hline
							Y &	T C (pirimidina/pYrimidine) \\\hline
						\end{tabular}
					\end{center}	
				\item Los códigos de aminoácidos soportados son:
					\begin{center}
						\begin{tabular}{| c | c |}
							\hline
							{\bf Código aminoácido} & {\bf Significado} \\
							\hline
							\hline		
							A &	Alanina \\\hline
							B & Ácido aspártico o Asparagina \\\hline
							C &	Cisteína \\\hline
							D &	Ácido aspártico \\\hline
							E &	Ácido glutámico \\\hline
							F &	Fenilalanina \\\hline
							G &	Glicina \\\hline
							H &	Histidina \\\hline
							I & Isoleucina \\\hline
							K &	Lisina \\\hline
							L &	Leucina \\\hline
							M & Metionina \\\hline
							N &	Asparagina \\\hline
							O &	Pirrolisina \\\hline
							P &	Prolina \\\hline
							Q &	Glutamina \\\hline
							R &	Arginina \\\hline
							S &	Serina \\\hline
							T &	Treonina \\\hline
							U &	Selenocisteína \\\hline
							V &	Valina \\\hline
							W &	Triptófano \\\hline
							Y &	Tirosina \\\hline
							Z &	Ácido glutámico o Glutamina \\\hline
							X &	cualquiera \\\hline
							* &	parada de traducción \\\hline
							- &	hueco (gap) de longitud indeterminada \\\hline
						\end{tabular}
					\end{center}	
			 \end{itemize}

		
		
\section{Bibliotecas existentes a utilizar}	
	\subsection{MiLi}
		\par \textsc{MiLi}\footnote{MiLi: \textcolor{blue}{mili.googlecode.com.}} es una colección de pequeñas y útiles librerías desarrolladas en el 			lenguaje de programación \textsc{C++} por \textbf{FuDePAN}, compuestas únicamente por cabeceras (\textit{headers}). No requiere instalación para 			su uso y ofrece soluciones simples para problemas sencillos.
		\par Provee varias funcionalidades mediante archivos cabecera, conocidos en el ámbito de C/C++ como archivos con extensión ``\textit{.h}''. 
		\par Dentro de las diversas funcionalidades podemos encontrar:
		\begin{itemize}
			\item \textbf{binary-streams:} permite serializar diferentes tipos de datos dentro de un único objeto utilizando los operadores de stream 											  ($\ll$ y $\gg$). Hay dos maneras de utilizar esta librería:
											\begin{enumerate}
												\item Empaquetar datos dentro de un objeto de salida (bostream) utilizando el operador $\prec \prec$.	
												\item Extraer datos desde un objeto de entrada (bistream) utilizando el operador $\succ \succ$.
											\end{enumerate}
			\item \textbf{container-utils:} esta biblioteca provee un conjunto de funciones, optimizadas para cada tipo de contenedor STL:
							\begin{itemize}
								\item \textsc{find(container, element)}.
								\item \textsc{find(container, element, nothrow)}.
								\item \textsc{contains(container, element)}.
								\item \textsc{insert into(container, element)}.
								\item \textsc{copy container(from, to)}.
							\end{itemize}
							\par Adicionalmente, esta biblioteca provee las siguientes clases:
							\begin{itemize}
								\item \textsc{ContainerAdapter$<$T$>$}.
								\item \textsc{ContainerAdapterImpl$<$T, ContainerType$>$}.
							\end{itemize}
						\par Estos container adapters son una herramienta para lidiar con diferentes contenedores \textsc{STL} de manera homogénea, sin 						necesidad de conocer el tipo de contenedor que es (vector, list, map, set). Los usuarios pueden invocar al método 							\textsc{insert (const T\&)} para insertar un elemento de tipo \textsc{T} sin la necesidad de saber el tipo del contenedor.

			\item \textbf{generic\_exception:} ofrece una implementación de excepciones genéricas a partir de las cuales los desarrolladores pueden crear 												sus propias excepciones para problemas específicos de una manera muy simple.
		\end{itemize}
		

	\subsection{FuD}
		\par \textsc{FuD}\footnote{\textbf{F}uDePAN \textbf{U}biquitous \textbf{D}istribution: \textcolor{blue}{fud.googlecode.com.}} es un framework para 			automatizar la implementación de aplicaciones distribuidas. 
		\par Este framework es un sistema separado en las capas de \textit{aplicación}, \textit{administración} y \textit{distribución} combinando los 			conceptos de cliente-servidor y Divide \& Conquer. Tanto el cliente como el servidor se encuentran organizados en estos tres niveles (muy) 			separados, cada uno de ellos con una única responsabilidad bien definida. La comunicación entre los diferentes niveles se encuentra estrictamente
		limitada, es decir, por cada nivel existe un único punto de comunicación ya sea para comunicarse con la capa superior o con la inferior. Cuando un 			mensaje es creado, éste debe atravesar las diferentes capas comenzando desde la de nivel más alto hacia la capa inferior, y luego recorrer en
		sentido contrario las capas del lado receptor.

		\subsubsection{Diseño de FuD}	
		\begin{itemize}
			\item \textsc{Application Layer (L3):} proporciona los componentes que contienen todos los aspectos del dominio del problema a resolver. Estos 														aspectos incluyen todas las definiciones de los datos usados y su manipulación correspondiente, como 														así también todos los algoritmos relevantes para la solución al problema en general. 
													\par Esta capa no es considerada como parte de FuD. La implementación de una aplicación que usa la 														librería, no es parte de la librería.
													\par Es necesario que del lado del servidor se implemente la aplicación principal, la cual hará uso de 														una simple interfaz en la abstracción de un trabajo distribuible permitiendo así codificar la 														estrategia de distribución de trabajos. Del lado cliente, solo se necesita implementar los métodos 														encargados de realizar las computaciones indicadas por una unidad de trabajo.
			\item \textsc{Job Management Layer (L2):} permite manejar los trabajos que se desean distribuir como así también generar las unidades de 														trabajo que serán entregadas a los clientes para su procesamiento. Estas unidades de trabajo llegan
													a su cliente correspondiente gracias a la capa más baja, encargada de la distribución. Una vez 														finalizado el procesamiento, se informa que todo ha terminado y otorga los resultados a la capa 													superior.
			\item \textsc{Distributing Middleware Layer (L1):} constituye un esquema de administración de clientes en particular. Tanto del lado cliente 																	como del servidor, la parte fija está dada por interfaces.	
																\par Las implementaciones concretas de este nivel son variables y están determinadas por
																el middleware a utilizar, por ejemplo \textsc{Boost.Asio}\footnote{\textcolor{blue}{http://www.boost.org/doc/libs/1\_4\_00/doc/html/boost\_asio.html}}, \textsc{MPI}\footnote{\textcolor{blue}{http://www.mcs.anl.gov/research/projects/mpi/}} o 
																\textsc{BOINC}\footnote{\textcolor{blue}{http://boinc.berkeley.edu/}}. 
		\end{itemize}	

	\subsection{BioPP}
		\par \textsc{BioPP}\footnote{\textcolor{blue}{biopp.googlecode.com.}} es una biblioteca C++ para Biología Molecular. La misma proveerá las 			diversas estructuras de datos y métodos para manipular las secuencias de nucleótidos con las que se trabajará. 		

	\subsection{Biopp-filer}
		 \par \textsc{Biopp-filer}\footnote{\textcolor{blue}{biopp-filer.googlecode.com.}} es una librería de persistencia para \textsc{Biopp}. 

	\subsection{Odf-gen}
		\textsc{Odf-gen}\footnote{\textcolor{blue}{odf-gen.googlecode.com.}} es una librería que ofrecerá funcionalidades que permiten generar archivos 			OpenDocument. Soporta diferentes tipos de archivos tales como:
		\begin{itemize}
			\item Hojas de cálculo. Compatible con OpenOffice Calc.
			\item Características compatibles: tablas, gráficos, gráficos de automóviles.
			\item APIs disponibles: C++, Python.
		\end{itemize}
		 \par Será utilizada para generar un archivo en el cual se almacenarán las tablas comparativas.	
		 
	\subsection{Fideo}
			\textsc{Fideo}\footnote{\textcolor{blue}{fideo.googlecode.com.}} es una librería proveerá, entre otras, las funcionalidades necesarias para 			obtener la energía libre de una secuencia de nucleótidos. Casi la totalidad de su código proviene del proyecto VAC-O\footnote{\textcolor{blue}{vac-o.googlecode.com}}. 						

\begin{thebibliography}{99}
\small	\bibitem{1} {\em{“FASTA:”}} \textcolor{blue}{http://en.wikipedia.org/wiki/FASTA\_format}

\small	\bibitem{2} {\em{“A Framework for Small Distributed Projects and a Protein Clusterer Application”}}
			\textsc{Biset, Guillermo}, 3 de diciembre de 2009.
\end{thebibliography}

\end{document}
