
\section{Conclusión}
      \begin{frame}\frametitle{\textbf{Resultados}}
        \begin{block}{Termodinámicos}
           \begin{itemize}
              \item \textit{Estructura Secundaria:} mayor cantidad de uniones GC en el humanizado que en el original. Mayor cantidad de stacks y más largos en el humanizado que en el original.

              \item \textit{$_m$$_i$RNA vs RNA$_m$:}$_m$$_i$RNA tienen una mayor capacidad de hibridar con los RNA humanizados que con los originales. 

            \end{itemize}                  
        \end{block}

        \begin{block}{Biológicos}
          \textbf{En los virus estudiados se pudo observar que existe un uso de codones divergente respecto de los del huésped humano.} En principio significaría una menor velocidad de síntesis proteica.
          Es decir, al virus le convendría usar los codones humanizados, pero sin embargo disminuye la cantidad de nucleotidos disponible para aparearse, los cuales estimulan la producción de interferón.
          %Desventaja: no tiene los codones opimos
          %ventaja: serviría para escapar al efecto del interferon

          %\par Surgieron además dos factores no contemplados en la biblografía (\emph{Futuros trabajos}).
        \end{block}
      \end{frame}    

      \begin{frame}\frametitle{\textbf{Conclusión}}
        \begin{block}{Profesional}
          \begin{itemize}
            \item Nuevos conceptos (SOLID, TMP)
            \item Profundización de conceptos 
            \item Exigencia en calidad y eficiencia
            \item Trabajo interdisciplinario como forma de aplicar los conocimientos científicos en la resolución concreta de problemas que afectan la calidad de vida de las personas.
          \end{itemize}
        \end{block}
        \begin{block}{Personal}
          \begin{itemize}
            \item Trabajo grupal
            \item Adaptación a un nuevo entorno de desarrollo
            \item Fuertes lazos humanos
          \end{itemize}
        \end{block}
      \end{frame}    

      \begin{frame}\frametitle{\textbf{Aportes}}
        \begin{itemize}
        \item A FuDePAN:
          \begin{itemize}
              \item R-emo: \url{r-emo.googlecode.com}.
              \item Fideo: \url{fideo.googlecode.com}.
              \item Acuoso: \url{acuoso.googlecode.com}.
              \item Etilico: \url{etilico.googlecode.com}.
              \item Otros: biopp(\url{biopp.googlecode.com}).
          \end{itemize}
        \vskip .2cm
        \item A la comunidad científica:
            \begin{itemize}                
                \item Análisis y la formalización del problema biológico.
                \item Solución computacional al problema.                
                \item Publicaciones (SAV 2012 - 3CAB2C). 
            \end{itemize}                 
         \vskip .2cm 
         \end{itemize}       
      \end{frame}    

      \begin{frame}\frametitle{\textbf{Trabajos Futuros}}
        \begin{block}{Futuros Desarrollos}
          \begin{itemize}
            \item Agregar nuevos backends para folding e hibridización en \emph{fideo}.
            \item Agregar nuevos backends para la humanización en \emph{acuoso}.
            \item Implementar un módulo de control de marco de lectura de las secuencias.
            \item \emph{Realizar el mismo estudia aplicado a proteínas.}
          \end{itemize} 
        \end{block}
        \begin{block}{Trabajos relacionados indirectamente}
          \begin{itemize}
            \item Pruebas de laboratorio.
            \item Refutación de las nuevas hipótesis.
          \end{itemize} 
        \end{block}
      \end{frame}    
