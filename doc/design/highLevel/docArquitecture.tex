\documentclass[12pt,a4paper]{article}

\usepackage[spanish,english]{babel}
\usepackage[utf8]{inputenc}

\title{Documentación de Arquitectura}
\author{Franco G. Riberi}

\begin{document}
\selectlanguage{spanish}

\section*{Introducción}
A continuación se describe brevemente cada uno de los componentes del sistema.

\section*{Catálogo de componentes}
\begin{itemize}
    \item \textbf{System:} corresponde a la aplicación en sí, la cual esta formada por los siguiente
        componentes:    
        \begin{itemize}
              \item \textbf{Main:} corresponde al módulo principal en términos de ejecución del sistema.

              \item \textbf{MasterOfPuppets (MOP):}  comprende la inicialización e invocación de los
              demás componentes. Coordina y gestiona la mayoría de las interacciones entre módulos.

              \item \textbf{GeneDesign:} representa el componente encargado de generar secuencias humanizadas.
              Dada una secuencia original, genera la secuencia humanizada correspondiente. Dicho
              módulo es externo a este desarrollo, y para ello se emplea el software \emph{GeneDesign}.

              \item \textbf{OutputsGenerator:} comprende la generación de archivos. Se crean tanto archivos
              como RNA$_m$ se analizen.

              \item \textbf{TableGenerator:} este componente es el encargado de rellenar las tablas. Para ello,
              realiza el matching por complemento y el cálculo de score entre secuencias de RNA$_m$ y 
              small-RNA$_s$. 

              \item \textbf{Validator:} encargado de controlar la calidad de las secuencias. Es el componente
              encargado de decidir, si una secuencia se encuentra en el marco de lectura correcto o no.
                
        \end{itemize}
    \item \textbf{Mili:} corresponde a una colección de funciones de C++, para resolver detalles de
     implementación.

    \item \textbf{fideo:} corresponde a una librería parcialmente ya implementada. Provee al sistema la
     funcionalidad de \emph{``folding''} e hibridación. 

    \item \textbf{Biopp:} biblioteca C++ para Biología Molecular. Permite la manipulación de secuencias de
    ácidos nucleicos.  

    \item \textbf{Biopp-filer:} para la lectura de secuencias en formato FASTA.    

    \item \textbf{getoptpp} para facilitar el manejo de entrada estándar.
\end{itemize}

\end{document}
