\section{Diseño de alto nivel}
\label{high_level_design}

\subsection{Interfaces - Responsabilidades - Colaboradores}
En esta sección se presentan las principales interfaces que intervienen en el sistema, sus respectivas responsabilidades y colaboradores. En la figura~\ref{interface} se exhibe el diagrama de interfaces correspondiente.

Finalmente, en la figura~\ref{mensajes} se presenta el diagrama de secuencia correspondiente a la comunicación entre las principales entidades del sistema.

\begin{figure}[!hbtp]
	\begin{center}
		\includegraphics[width=15.5cm,height=11cm]{interface.png}
		\caption{UML - Interfaces}
		\label{interface}
	\end{center}
\end{figure}

\subsubsection{IValidator}
\par \textbf{Responsabilidad:} Realizar el control de calidad para las secuencias. Determinar si una secuencia aprueba o no el control de calidad. 

%biopp
\par \textbf{Colaboradores:}
 
\subsubsection{IFold}
\par \textbf{Responsabilidad:} Proveer al sistema el \emph{``folding''} de secuencias de RNA.

\par \textbf{Colaboradores:} \\
\hspace*{3.75cm} 1. Vienna Package (RNAFold). \\
\hspace*{3.75cm} 2. UNAFold. \\
\hspace*{3.75cm} 3. MFold. \\
\hspace*{3.75cm} 4. Otros.

\subsubsection{ISequence}
\par \textbf{Responsabilidad:} Provee al sistema el \emph{``matching''} de secuencias.
	
\par \textbf{Colaboradores:} 

\subsubsection{IData}
\par \textbf{Responsabilidad:} Provee al sistema las secuencias de small-RNA$_s$.

\par \textbf{Colaboradores:} \\
\hspace*{3.75cm} 1. DataBase \\
\hspace*{3.75cm} 2. BLAST \\

\begin{figure}
  \centering
  \includegraphics[scale=0.5, angle=90]{diagramaDeSecuencias.png}  
  \caption{UML - Pasaje de mensajes}
  \label{mensajes}
\end{figure}
