\chapter{Conclusiones del Trabajo}

\epigraph{Pones tu pie en el camino y si no cuidas
tus pasos, nunca sabes a donde te pueden
llevar.}%
{\textbf{John Ronald Reuel Tolkien}}

\section{Resultados}
\textcolor{red}{complete}
Informar los datos con que se corrio finalmente

\section{Conclusión Final}
\textcolor{red}{complete}

\section{Logros}
Este proyecto de trabajo final fue presentado previamente en dos oportunidades:
\begin{itemize}
    \item \emph{\textbf{Relationship between divergence of using synonymous codons in host-virus and the presence of microRNA.}} Franco	Riberi, Laura Tardivo, Lucia Fazzi, Guillermo Biset, Daniel Gutson, Daniel Rabinovich. En 3CAB$^{2}$C (3er. Congreso Argentino de Bioinformática y Biología Computacional).

    \item \emph{\textbf{Estudio de la relación entre divergencia  en el uso de codones del virus respecto al huésped y reconocimiento por los microRNA.}} Riberi F, Fazzi L, Tardivo L, Gutson D, Rabinovich D. En XXXII Reunión Científica Anual (SAV2012-Sociedad Argentina de Virología).    
\end{itemize}

Además, \textbf{Remo} fue abalado por Secretaria de Ciencia y Técnica de la UNRC otorgando la beca TICs para la finalización de carrera.

\section{Aportes}
\par A FuDePAN:
\begin{itemize}
    \item Fideo   : \url{fideo.googlecode.com}
    \item Acuoso  : \url{acuoso.googlecode.com}
    \item Etilico : \url{etilico.googlecode.com}
    \item Otras   : aportes en otras librerias tales como mili\url{mili.googlecode.com} y biopp\url{biopp.googlecode.com}.
\end{itemize}

\par A la comunidad científica:
\par El principal aporte de este trabajo fue el análisis y la formalización del
problema biológico y la posterior propuesta de cómo solucionarlo
computacionalmente. 

\par Por otra parte, de alguna manera esto pretende ser un aporte al trabajo interdisciplinario como forma de aplicar los conocimientos científicos en la resolución concreta de problemas que afectan la calidad de vida de las personas.

\section{Trabajos Futuros}
\begin{itemize}
	\item Agregar nuevos backends para folding en \emph{fideo}.
	\item Agregar nuevos backends para hibridización en \emph{fideo}.
	\item Agregar nuevos backends para la humanización en \emph{acuoso}.
	\item Implementar un módulo de control de marco de lectura de las secuencias.
	\item Realizar el mismo estudia aplicado a proteínas.
\end{itemize}

\section{Repositorio del sistema}
Se puede acceder al código fuente de este proyecto y a su documentación visitando \url{http://r-emo.googlecode.com}.


