\chapter{Conclusiones del Trabajo}

\epigraph{Pones tu pie en el camino y si no cuidas
tus pasos, nunca sabes a donde te pueden
llevar.}%
{\textbf{John Ronald Reuel Tolkien}}

\section{Resultados Obtenidos}
\par Determinar si la divergencia en el uso de codones sinónimos entre virus-huésped contribuye a disminuir la interferencia de los $_m$$_i$RNA en la expresión de los mensajeros de origen viral puede ser abordado desde tres enfoques diferentes: análisis de estructuras secundaria, $_m$$_i$RNA vs RNA$m_$ y proteínas. En el presente trabajo se abordaron los dos primeros enfoques.

\begin{itemize}
    \item Desde \emph{el punto de vista termodinámico:}
        \begin{itemize}
        \item \textbf{A nivel estructura secundaria:} se obtuvo que existe mayor cantidad de uniones GC en el humanizado que en el original, y viceversa, mayor cantidad de uniones AT en el original, que en el humanizado. 
        \par Las uniones GC son más fuertes que las AT, porque entre estos nucleótidos existe una mayor fuerza de atracción que en la unión AT. En un análisis estructural es esperable encontrar que existan más stacks (uniones) y de mayor tamaño, en la secuencia humanizada que en la original. A pesar de encontrar estas cantidades de nucleótidos, las secuencias humanizadas tienen mayor desproporción o desbalance entre GC y entre AT. Esto conduciría a que ciertos nucleótidos no puedan aparearse.

        \par Como resultado de la comparación de estructuras secundarias se obtuvo que existe mayor cantidad de stacks en el humanizado que en el original.

        \item \textbf{A nivel $_m$$_i$RNA vs. RNA$_m$:} la comparación muestra que en general los $_m$$_i$RNA tienen una mayor capacidad de hibridar con los RNA humanizados que con los originales. Este análisis es genérico para todo el genoma, donde se barre todo el gen. Por otro lado, si sólo se rescatan los valores máximos de score\footnote{Score hace referencia a cantidad de uniones posibles.} tanto para original como humanizado, se observa el mismo resultado en la mayoría de los casos.
        \end{itemize}
    \item Desde \emph{el punto de vista biológico:} \textbf{en los virus estudiados existe un uso de codones divergente respecto de los del huésped humano}. En principio significaría una menor velocidad de síntesis proteica y estos virus estarían en desventaja en la replicación intracelular y de no existir otros factores serían desplazados por mutantes cuyo uso de codones sea más semejante al humano.

    \par Se intentaron varias explicaciones para este hecho. Los virus podrían utilizar esta divergencia como un método de regulación fina de la síntesis proteica o podrían encontrar menor competencia por los RNA$_t$ (RNA de transferencia) utilizados por el RNA$_m$ del huésped. Los resultados demostrados sugieren dos factores alternativos no contemplados en la bibliografía:

    \begin{enumerate}
    \item El RNA viral presente en la naturaleza tiene menos stacks que simulan una cadena doble del RNA que el supuesto virus humanizado. El RNA de doble cadena es característico de los virus, es reconocido por receptores de las células como el MID5 y el TLR 3  (Toll like receptor 3), este reconocimiento desencadena una serie de reacciones que inducen la síntesis de interferón, una proteína de intensa acción antiviral.

    \item El RNA viral presente en la naturaleza tiene menos capacidad de ser reconocidos por $_m$$_i$RNA codificados por el huésped que el potencial virus humanizado. Los $_m$$_i$RNA son reguladores negativos de la síntesis proteica y por lo tanto este factor podría compensar el uso menos efectivo de los RNA$_t$.
    \end{enumerate}

\end{itemize}

\par En conjunto, estos resultados aportan a entender la importancia de un uso determinado de codones en la replicación del virus. Tienen además, un potencial uso en el desarrollo de vacunas y en la síntesis de proteínas recombinantes en células eucariotas. 


\par Cabe destacar que en lo mencionado anteriormente se empleó un vocabulario técnico virológico/biológico propio de ambas áreas. La tarea de interpretar los resultados de este trabajo desde el punto de vista computacional puede ser muy dificultoso, dado que se empleó la computación como una herramienta, pero no por ello de menor importancia. El potencial de las ciencias de la computación dió como resultado una respuesta a un problema real, el cual fue especificado y abordado mediante el trabajo interdisciplinar.  

\par En términos generales, se puede decir que no sólo que se logró contrastar la hipótesis planteada inicialmente, sino que surgieron dos nuevas hipótesis a analizar en futuras extensiones del presente trabajo.

\section{Conclusión Final}
\par Como la mayoría de la actividad humana, el trabajo científico y profesional está enmarcado por las necesidades e ideas del hombre. Dichas ideas requieren de mucho esfuerzo para poder ser llevadas a cabo con éxito, como así también surgen diversos contratiempos y dificultades que oscurecen el punto final, o quizás el principio de algo inesperado, dado que se pueden abrir nuevos caminos no conocidos hasta entonces.

\par Transitar el proceso que desembolsó en el presente trabajo fue continuamente desafiante, tanto desde el punto de vista humano como profesional.

\par En lo que respecta a lo profesional, permitió el trabajo interdisciplinar acoplando diferentes ciencias con un objetivo en común. Se formalizó un problema real de índole biológico. Además se adquirieron nuevos conceptos, referentes tanto al dominio del problema como también relacionados al desarrollo de software, tales como el acrónico SOLID, diseño orientado a responsabilidades, depuración, entre otros. Por otro lado se experimentó el desarrollo distribuido con diversas personas situadas en distintos puntos geográficos, se aplicaron diversos conceptos teóricos estudiados durante la carrera, tales como el uso de patrones de diseño en un contexto real, ingeniería inversa, etcétera. 

\par Desde el punto de vista personal, se adquirió gran destreza para trabajar sobre código ajeno, detección, reporte y resolución de bugs, manejo de software para la gestión y seguimiento de proyectos, etcétera. 

\par Personalmente, en mi carácter de autor del presente escrito me gustaría destacar la calidez humano de las personas que trabajaron a mi lado, tanto de forma directa como indirecta. Pienso que esta conclusión no es un cierre a este trabajo, sino una puerta que se abre, quedando a pie de cañón realizar el mismo estudio a nivel proteína, lo cual aumenta aún más la complejidad dada la estructura de las mismas.

\section{Logros}
Este proyecto de trabajo final fue presentado previamente en dos oportunidades:
\begin{itemize}
    \item \emph{\textbf{Relationship between divergence of using synonymous codons in host-virus and the presence of microRNA.}} Franco	Riberi, Laura Tardivo, Lucia Fazzi, Guillermo Biset, Daniel Gutson, Daniel Rabinovich. En 3CAB$^{2}$C (3er. Congreso Argentino de Bioinformática y Biología Computacional).

    \item \emph{\textbf{Estudio de la relación entre divergencia  en el uso de codones del virus respecto al huésped y reconocimiento por los microRNA.}} Riberi F, Fazzi L, Tardivo L, Gutson D, Rabinovich D. En XXXII Reunión Científica Anual (SAV2012-Sociedad Argentina de Virología).    
\end{itemize}

Además, \textbf{Remo} fue abalado por Secretaria de Ciencia y Técnica de la UNRC otorgando la beca TICs para la finalización de carrera.

\section{Aportes}
\par A FuDePAN:
\begin{itemize}
    \item Remo   : \url{r-emo.googlecode.com}
    \item Fideo   : \url{fideo.googlecode.com}
    \item Acuoso  : \url{acuoso.googlecode.com}
    \item Etilico : \url{etilico.googlecode.com}
    \item Otras   : aportes en otras librerías tales como mili\url{mili.googlecode.com} y biopp\url{biopp.googlecode.com}.
\end{itemize}

\par A la comunidad científica:
\par El principal aporte de este trabajo fue el análisis y la formalización del
problema biológico y la posterior propuesta de cómo solucionarlo
computacionalmente. 

\par Por otra parte, de alguna manera esto pretende ser un aporte al trabajo interdisciplinario como forma de aplicar los conocimientos científicos en la resolución concreta de problemas que afectan la calidad de vida de las personas.

\section{Trabajos Futuros}
\begin{itemize}
	\item Agregar nuevos backends para folding e hibridización en \emph{fideo}.	
	\item Agregar nuevos backends para la humanización en \emph{acuoso}.
	\item Implementar un módulo de control de marco de lectura sobre las secuencias.
	\item \emph{Realizar el mismo estudia aplicado a proteínas.}
\end{itemize}

\section{Repositorio del sistema}
Se puede acceder al código fuente de este proyecto y a su documentación visitando \url{http://r-emo.googlecode.com}.


