\chapter{Detalles de FuD}
\label{detFud}

\par \textbf{FuD} fue implementado empleando usando un esquema \emph{Divide \& Conquer} donde el procesamiento es llevado a cabo por los nodos procesadores. ``Divide'' cuando se realiza la división de un trabajo en una unidad de trabajo. ``Conquer'' al incorporar los resultados de una unidad de trabajo.

\section{Diseño}
A continuación se describe brevemente cada una de las capas que constituyen el framework.

\subsection{Application Layer (L3)} 
\par Básicamente proporciona los componentes que contienen todos los aspectos del dominio del problema a resolver. Estos aspectos incluyen todas las definiciones de los datos usados y su manipulación correspondiente, como así también todos los algoritmos relevantes para la solución al problema en general. 

\par Es necesario que del lado del servidor se implemente la aplicación principal, la cual hará uso de una simple interfaz en la abstracción de un trabajo distribuible permitiendo así codificar la estrategia de distribución de trabajos. Del lado cliente, solo se necesita implementar los métodos encargados de realizar las computaciones indicadas por una unidad de trabajo.

\subsection{Job Management Layer (L2)} %manejador de trabajo
\par Esta capa permite manejar los trabajos que se desean distribuir como así también generar las unidades de trabajo que serán entregadas a los clientes para su procesamiento. Estas unidades de trabajo llegan a su cliente correspondiente gracias a la capa más baja, encargada de la distribución. Una vez finalizado el procesamiento, se informa que todo ha terminado y otorga los resultados a la capa superior.

\subsection{Distributing Middleware Layer (L1)} %comunicación
\par En esta capa existe el único vínculo real entre clientes y servidor. La responsabilidad principal es manejar los clientes conectados al servidor y llevar a cabo los procedimientos de comunicación entre ambos.

\par Las implementaciones concretas de este nivel son variables y están determinadas por el middleware a utilizar, por ejemplo \textsc{Boost.Asio}\footnote{\url{http://www.boost.org/doc/libs/1\_4\_00/doc/html/boost\_asio.html}}, \textsc{MPI}\footnote{\url{http://www.mcs.anl.gov/research/projects/mpi/}} o \textsc{BOINC}\footnote{\url{http://boinc.berkeley.edu/}}. 
		
\par Actualmente, \emph{FuD} cuenta con dos capas más, conocidas como \emph{FuD-RecAbs} y \emph{FuD-CombEng}. Estas capas no se describen dado que sólo se empleará \textbf{FuD} original, para mayor información consultar \url{fud.googlecode.com.}