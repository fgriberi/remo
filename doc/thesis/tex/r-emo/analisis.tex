\chapter{Elicitación y Análisis de Requerimientos}
\label{analisis}

\epigraph{No hay enigmas, si un problema puede plantearse, 
es que puede resolverse.}%
{\textbf{Ludwig Wittgenstein}}

\par El presente capítulo describe de forma general la elicitación de requerimientos, y de manera más detallada el análisis de los mismos.

\section{Elicitación de Requerimientos}
\par La IEEE\footnote{\url{www.ieee.org}} define los requerimientos como: 
\begin{center}
``Una condición de la capacidad que necesita un usuario para resolver un problema o lograr un objetivo. Una condición o una capacidad que debe cumplir un sistema para satisfacer un contrato, un estándar, una especificación, o cualquier otro documento oficial establecido.''
\end{center}
Los requerimientos de software del presente trabajo fueron provistos por miembros de \textbf{FuDePAN} y quedaron documentados en la ``Especificación de Requerimientos de Software'' (SRS) que se encuentra en el repositorio del proyecto. Dicho documento describe qué hará el software propuesto pero no cómo lo hará.

\subsection{Requerimientos Funcionales}
Los requerimientos funcionales de \remo pueden resumirse de la siguiente manera:

\begin{itemize}
	\item \textbf{Entrada:} dos archivos en formato FASTA (ver Apéndice~\ref{fasta}). El primero de ellos conteniendo un conjunto de secuencias de $_m$$_i$RNA, y el segundo, compuesto por múltiples secuencias de $_m$RNA.

	\item \textbf{Objetivo:} satisfaciendo las restricciones impuestas, determinar la relación de divergencia y contrastar la hipótesis plasmada anteriormente (luego de un extenso análisis biológico de los resultados). En este ámbito uno de los requisitos más importantes fue calcular la estructura secundaria de las secuencias de RNA (folding/hibridación).
							 
	\item \textbf{Salida:} archivos en formato CSV, que permitan poder establecer conclusiones al respecto. 
\end{itemize}

\par Uno de los principales requerimientos funcionales que cabe destacar tiene relación con la comparación de estructuras secundarias. Para ello, por cada RNA de entrada, es necesario calcular la estructura secundaria. A partir de la misma, realizar un reconocimiento de los diversos \emph{motif} (ver sección ~\ref{RNAmotifs}) que la componen para generar como salida un archivo comparativo. Las estructuras pueden ser muy grandes y complejas, tales como la que se exhibió en la figura~\ref{motifs}.

\par Desde el punto de vista del diseño, \remo debía cumplir con los principios de diseño conocido por el acrónico SOLID\footnote{Acrónimo nemotécnico introducido por Robert C. Martin en la década del 2000, que representa cinco principios básicos de la programación y diseño orientado a objetos}~\cite{Martin00}. Estos principios son los siguientes y se retomaran el la sección~\ref{disenio}:
\begin{itemize}
	\item \textbf{S}ingle responsibility principle (SRP)
	\item \textbf{O}pen/closed principle (OCP)
	\item \textbf{L}iskov substitution principle (LSP)
	\item \textbf{I}nterface segregation principle (ISP)
	\item \textbf{D}ependency inversion principle (DIP)	
	\item Law of Demeter (LoD)
\end{itemize}

\par Para observar detalladamente los requerimientos funcionales puede consultarse el SRS\footnote{\url{r-emo.googlecode.com}} correspondiente a \remo.

\section{Análisis de los Requerimientos}
\par El objetivo básico del análisis de problemas es obtener una comprensión clara de las necesidades de los clientes, en otros término, qué es exactamente lo que se desea del software, y cuales son las limitaciones de la solución. El principio elemental empleado en el análisis es el mismo que en cualquier tarea compleja: \emph{divide y vencerás}. Esto es la “partición” del problema en subproblemas y a continuación, tratar de entender cada subproblema y su relación con otros subproblemas en un esfuerzo por entender el problema total.

\par Una vez identificados los diversos requerimientos, se realizó un extenso análisis referente a las herramientas a emplear y al dominio del problema en general. A continuación se detalla dicho análisis.

\subsection{Búsqueda de datos mínimos}
	\par Esta etapa consistió en la recolección de datos necesarios para posteriormente realizar las primeras pruebas de \textbf{Remo}. Esta fase involucró la recolección de secuencias de $_m$RNA, secuencias de $_m$$_i$RNA y el software humanizador de secuencias. 
	
\subsubsection{$_m$RNA}
	\par La bases de datos biológicas son DNA céntricas, es decir que si un virus tiene su información como RNA, la secuencia archivada esta expresada como DNA (\textbf{T}imina en lugar de \textbf{U}racilo). Por lo cual, no se encontró $_m$RNA directamente, sino que como gen.

	\par Los virus obtenidos para las pruebas se obtuvieron de \\
	~\cite{holme}. Dicha publicación menciona el número de acceso a \emph{GenBank}\footnote{Base de datos de secuencias genéticas. Colección de todas las secuencias de DNA disponibles al público. La base de datos GenBank está diseñada para proporcionar y fomentar el acceso de la comunidad científica a la mayor parte de información actualizada y completa la secuencia de DNA. Por lo tanto, \textit{NCBI} (National Center for Biotechnology Information \url{http://www.ncbi.nlm.nih.gov/}) no impone restricciones sobre el uso o la distribución de los datos GenBank.} para su obtención. A continuación se exhiben algunos de los códigos que se mencionan el artículo: 
	\begin{itemize}
		\item Coxsackievirus A9 (Griggs; \textsc{D00627}).
		\item Enterovirus 71 (TW/2272/98; \textsc{AF119795}).
		\item Hepatitis A virus (HAF-203; \textsc{AF268396}).
		\item Poliovirus type 3 (23127; \textsc{X04468}).
		% \item Rhinovirus type 89 (HRV89; \textsc{M16248}).
		% \item Hepatitis E virus (Hyderabad; \textsc{AF076239}).
		% \item Norwalk virus (BS5; \textsc{AF093797}).
		% \item Astro-virus type 1 (Oxford; \textsc{L23513}).
		\item Dengue-1 virus (PDK-13; \textsc{AF180818}).
		% \item Dengue-2 virus (S1; \textsc{NC\_001474}).
		% \item Dengue-3 virus (H87; \textsc{M93130}).
		% \item Dengue-4 virus (\textsc{NC\_002640}).
		% \item GB virus C (G05BD; \textsc{AB003292}).
		% \item Hepatitis C virus (HCV-A; \textsc{AJ000009}).
		% \item Japanese encephalitis virus (GP78; \textsc{AF075723}).
		% \item Murray Valley virus (MVE-1-51; \textsc{AF161266}).
		% \item West Nile virus (2741; \textsc{AF206518}).
		% \item Western tick-borne encephalitis virus (Neudoerfl; \textsc{U27495}).
		\item Yellow fever virus (Trinidad 79A; \textsc{AF094612}).
		% \item Eastern equine encephalitis virus (\textsc{U01034}).
		% \item O’nyong-nyong virus (SG650; \textsc{AF079456}).
		% \item Ross River virus (NB5092; \textsc{M20162}).
		\item Rubella virus (Cendehill; \textsc{AF188704}).
		% \item Sindbis virus (Edsbyn 82-5; \textsc{M69205}).
	\end{itemize}

	\par A partir de los códigos de acceso, desde \url{http://www.ncbi.nlm.nih.gov/genbank/} se obtuvieron los DNA de cada virus en formato FASTA. Para traducir los DNA a RNA (reemplzado de \textsc{U} por \textsc{T}), se utilizó la herramienta \emph{BioEdit}\footnote{\url{http://www.mbio.ncsu.edu/BioEdit/bioedit.html}}. 

	% Alternativamente, se usó de la herramienta \emph{Jalview}\footnote{\url{http://www.jalview.org/}} y la librería \textit{Biopp}\footnote{\url{biopp.googlecode.com}}. 

\subsubsection{$_m$$_i$RNA}
	\par Se obtuvo una base de datos\footnote{Descargada de \url{http://www.mirbase.org/cgi-bin/mirna\_summary.pl?org=hsa}} de $_m$$_i$RNA en formato FASTA. De la misma, se seleccionaron al azar aproximadamente 50 $_m$$_i$RNA para luego realizar las primeras pruebas. Los $_m$$_i$RNA están formados por 22 nucleótidos aproximadamente. A continuación se exhiben algunas de las secuencias: 
\begin{itemize}
	\item \textsc{AAGAUGUGGAAAAAUUGGAAUC}
	\item \textsc{CAGUGGUUUUACCCUAUGGUAG}
	\item \textsc{CAGGCAGUGACUGUUCAGACGUC}
	\item \textsc{AUCAUAGAGGAAAAUCCAUGUU}
	\item \textsc{AAGAAGAGACUGAGUCAUCGAAU}
	\item \textsc{CUCUAGAGGGAAGCGCUUUCUG}
	\item \textsc{GUUUGCACGGGUGGGCCUUGUCU}
	\item \textsc{AAAGACCGUGACUACUUUUGCA}
	\item \textsc{UGUAAACAUCCCCGACUGGAAG}
\end{itemize}
		
	\par Otras bases de datos de $_m$$_i$RNA consultadas fueron: 
		\par \hspace*{1cm} • \url{http://micrornadatabase.com/}.
		\par \hspace*{1cm} • \url{http://microrna.org/}.
		\par \hspace*{1cm} • \url{http://mirdb.org/miRDB/}.
		\par \hspace*{1cm} • \url{http://www.mirbase.org/}.

\subsection{Herramientas y Librerías}
\subsubsection{GeneDesign}
\emph{GeneDesign}\footnote{Descarga de: \url{https://github.com/GeneDesign/GeneDesign}}~\cite{geneDesign} fue el software empleado para la ``humanización'' de las cadenas de nucleótidos. El mismo, corresponde a un software libre usado por biólogos moleculares, el gobierno y los sectores farmacéuticos, químicos, agrícolas y biotecnológicos para el diseño~\cite{Welch}, clonación y validación de secuencias genéticas.

\par \emph{GeneDesign} automatiza el proceso de determinar qué pares de bases deben ser unidos entre sí en un orden determinado para un gen. Un gen codifica para una proteína específica, y el orden de los cientos o miles de pares de bases que componen ese gen determina el orden de los bloques de construcción de aminoácidos que componen esa proteína.  

\par Este software, consta de seis módulos que pueden ser utilizados individualmente o en serie para automatizar las tareas necesarias para diseñar y manipular secuencias de DNA sintético. 

\par Se presentaron tres opciones para el uso de este software:
	\begin{itemize}
		\item \emph{Instalado:} corresponde a tener \emph{geneDesign} instalado en la PC, y el producto a desarrollar lo invocará a medida que lo requiera.
		\item \emph{Online:} el producto a desarrollar utiliza \emph{geneDesign} de forma online a medida que lo necesita. (\url{http://genedesign.thruhere.net/gd})
		\item \emph{Offline:} correr de forma online \emph{geneDesign} y generar con su salida dos archivos en formato FASTA. Uno de ellos, con el $_m$RNA original, y el otro con el humanizado. Luego el producto a desarrollar tomará los archivos mencionados como entrada.
	\end{itemize}
	
\par \remo empleó la opción número 1 (\emph{Instalado}) otorgando de esta forma mayor independencia y evitando la necesidad de conexión a internet para obtener los resultados. Para esta opción fue necesario descargar el código de \emph{geneDesign} del repositorio git e instalar y configurar el servidor web \emph{apache2}\footnote{url{http://httpd.apache.org/}}.

\par Cabe destacar que se encontró un bug en el código, básicamente el software siempre realizaba la conexión al servidor central por lo cual, aunque se ejecutara localmente con el servidor apache, no funcionaba sin conexión a internet, dado que siempre se establecía un socket con el servidor central.

\subsubsection{Librerías a Utilizar}
Además de las librería estándares que provee el jugoso lenguaje de programación C++, cabe mencionar las siguientes:
\begin{itemize}
	\item \emph{MiLi\footnote{\url{mili.googlecode.com}}}: es una colección de pequeñas y útiles librerías desarrolladas en el 
	lenguaje de programación C++ por \textbf{FuDePAN}, compuestas únicamente por cabeceras (\emph{headers}). No requiere
	instalación para su uso y ofrece \emph{soluciones simples para problemas sencillos}.
		 
	\par Dentro de las diversas funcionalidades podemos encontrar:
	\begin{itemize}
		\item \textbf{binary-streams:} permite serializar diferentes tipos de datos dentro de un único objeto utilizando los
		 operadores de stream ($\ll$ y $\gg$). Hay dos maneras de utilizar esta librería:
			\begin{enumerate}
				\item Empaquetar datos dentro de un objeto de salida (bostream) utilizando el operador $\ll$.	
				\item Extraer datos desde un objeto de entrada (bistream) utilizando el operador $\gg$.
			\end{enumerate}
		\item \textbf{container-utils:} esta biblioteca provee un conjunto de funciones, optimizadas para cada tipo de 					contenedor STL:
				\begin{itemize}
					\item \textsf{find(container, element)}.
					\item \textsf{find(container, element, nothrow)}.
					\item \textsf{contains(container, element)}.
					\item \textsf{insert into(container, element)}.
					\item \textsf{copy container(from, to)}.
				\end{itemize}
				\par Adicionalmente, esta biblioteca provee las siguientes clases:
				\begin{itemize}
					\item \textsf{ContainerAdapter$<$T$>$}.
					\item \textsf{ContainerAdapterImpl$<$T, ContainerType$>$}.
				\end{itemize}
				\par Estos container adapters son una herramienta para lidiar con diferentes contenedores \textsc{STL} de manera 					homogénea, sin necesidad de conocer el tipo de contenedor que es (vector, list, map, set). Los usuarios pueden 					invocar al método \textsc{insert (const T\&)} para insertar un elemento de tipo \textsc{T} sin la necesidad de 					saber el tipo del contenedor.

		\item \textbf{generic\_exception:} ofrece una implementación de excepciones genéricas a partir de las cuales los 			desarrolladores pueden crear sus propias excepciones para problemas específicos de una manera muy simple.
	\end{itemize}
		
	\item \emph{Biopp\footnote{\url{biopp.googlecode.com}}}: corresponde a una biblioteca C++ para Biología Molecular. La misma
	 provee las diversas estructuras de datos y métodos para manipular las secuencias de nucleótidos con las que se trabajó. 

	\item \emph{Biopp-filer\footnote{\url{biopp-filer.googlecode.com}}}: refiere a una librería de persistencia para 
	\emph{Biopp}. Esta librería se utilizó para leer las secuencias en formato FASTA.

	\item \emph{Getoptpp\footnote{\url{getoptpp.googlecode.com}}}: es una librería que proveerá entre otras cosas, diversas
	 funciones para el manejo de la entrada estándar.
\end{itemize}

\subsection{Librerías a Implementar}
\label{libreriaAImplementar}

\par Como resultado del análisis y pensando en la correcta modularización y futuros desarrollos, surgió la necesidad de implementar las siguientes librerías: 

\begin{itemize}
	\item \textbf{fideo (Folding Interface Dynamic Exchange Operations): } el problema que se presentó radica en que cada librería externa tanto para folding como para hibridación, poseen diferentes formas de recibir los parámetros de entrada y diferentes formas de retornar los resultados que generan. A razón de esto se decidió implementar \emph{fideo} como una forma de unificar el acceso a estas librerías externas e integrarlas al resto del sistema. Básicamente, se encargará de proveer las funcionalidades necesarias para obtener la energía libre de una secuencia de nucleótidos. Esta librería se utilizará para la invocación externa de los programas de cálculo de estructura secundaria (folding e hibridación). Por un lado, incluye los backends UNAFold y RNAFold para folding, y por el otro, incluye los backends RNAup, RNAcofold, RNAduplex, IntaRNA y RNAHybrid para la hibridación. Fideo provee la abstracción necesaria para poder utilizar indistintamente cualquiera de los backends mencionados de forma transparente.

	\item \textbf{acuoso (Abstract Codon Usage Optimization Software for Organisms): } permite la optimización (humanización) de secuencias. La idea es similar a fideo, dado que permite realizar este servicio con cualquier software de manera muy simple. Inicialmente se implementó empleando \emph{GeneDesign}, pero de manera sencilla puede agregarse cualquier otro backend.

	\item \textbf{etilico (External Tools Invocation LIbrary and COmponents)}: básicamente corresponde a una librería estática compuesta por diversas funciones necesarias y de uso común para invocación de librerías, manejo de archivos temporales y diversos componentes.
\end{itemize}

\subsection{Backends para Folding}

\subsubsection{UNAFold}
\label{unafold}
\par Inspirado en el famoso software ``mfold'', el paquete de software \emph{UNAFold}~\cite{unafold} es un conjunto integrado de programas (empleando programación dinámica~\cite{ProgDinaminca}) que simulan el folding, la hibridación y la vías de fusión para una o dos secuencias de cadena simple de ácido nucleico. El paquete predice el folding para RNA de cadena simple o el DNA a través de la combinación de minimización de la energía libre, los cálculos de función de partición y muestreo estocástico. El nombre se deriva de ``Folding Unificado de Ácido Nucleico''. Acepta tanto archivos planos (que contengan únicamente nucleótidos) como en formato FASTA. Está disponible para su descarga en \url{http://www.bioinfo.rpi.edu/applications/hybrid/download.php.}
A continuación se exhibe el cálculo de folding de una secuencia de prueba.

{\scriptsize
	\begin{verbatim}
        gringusi@gringusi:~$ echo "AAAAAAAAGGGGGGGGCCCCCCCCTTTTTTTT" > seq.fasta
        gringusi@gringusi:~$ UNAfold --max=1 seq.fasta

        Checking for boxplot_ng... not found
        Checking for hybrid-plot-ng... found, supports Postscript
        Checking for sir_graph_ng or sir_graph... not found
        Checking for ps2pdfwr... found
        Calculating for seq.fasta, t = 37
        gringusi@gringusi:~$ l
        seq.fasta seq.fasta_1.ct seq.fasta.ann seq.fasta.ct seq.fasta.det seq.fasta.dG 
        seq.fasta.h-num seq.fasta.plot seq.fasta.rnaml seq.fasta.run seq.fasta.ss-count
    \end{verbatim}
}

Dentro de los diferentes archivos generados se detacan el \emph{.dG}, \emph{.ct} y \emph{.det}. El primero de ellos contiene el valor de la energía libre en la unidad kcal/mol. El segundo,  define una secuencia de ácido nucleico, junto con su estructura secundaria. Puede contener pliegues múltiples de una secuencia única o incluso múltiples plegamientos. Por último, el tercero, brinda la estructura secundaria en un formato particular, donde se pueden identificar los diversos \emph{motifs} que la componen.

\subsubsection{RNAFold}
\label{rnafold}
\par Básicamente lee secuencias de RNA de entrada estándar y calcula el mínimo de energía libre (MFE), la función de partición (pf) y una matriz de probabilidad en base al apareamiento. Devuelve como salida la estructura mfe en notación soporte (empleando ``()''), su energía, la energía libre del conjunto termodinámico y la frecuencia de la estructura mfe. También produce archivos PostScript con parcelas de la gráfica resultante estructura secundaria y un ``gráfico de puntos'' de la matriz de apareamiento de bases. Para comprender mejor su comportamiento y método consultar~\cite{vienna}.

\par Está disponible para su descarga en \url{http://mfold.rna.albany.edu/?q=DINAMelt/software.}

{\scriptsize
	\begin{verbatim}
        gringusi@gringusi:~$ RNAfold

        Input string (upper or lower case); @ to quit
        ....,....1....,....2....,....3....,....4....,....5....,....6....,.
        AAAGGCAACGGCCAU
        length = 15
        AAAGGCAACGGCCAU
        ...(((....))).. 
        minimum free energy =  -4.40 kcal/mol
	\end{verbatim}
}

\par El resultado obtenido es, precisamente, la energía libre y la estructura secundaria representada con paréntesis y puntos, donde los pares de paréntesis indican las bases ``apareadas'' o ``unidas'' y los puntos, las bases libres.

\subsection{Backends para Hibridación} %mirar bien referencias
\label{hibrid}

\subsubsection{IntaRNA}
\par \emph{IntaRNA~\cite{intaRNA}} predice interacciones entre dos moléculas de RNA, por ejemplo, un RNA no codificante (e.g
ncRNA) y un mRNA. El scoring se basa en la energía combinada de interacción que resulta de la suma de la energía libre de hibridación y la energía libre necesaria para la fabricación de los sitios de interacción accesible de ambas moléculas.

\par La interacción tiene que contener la semilla de una interacción, es decir, una región (casi) perfecta de complementariedad para facilitar el inicio de la interacción. Las características de esta región de semilla son definidas por el usuario.

\par La accesibilidad se define como la energía libre necesaria para desplegar el sitio de interacción en cada molécula. Requiere del paquete de Vienna RNA (versiones 1.8.2/1.8.5). La versión 1.8.5 puede descargarse de \url{http://www.bioinf.uni-freiburg.de/Software/index.html?en\#IntaRNA-download}

\par Existen dos variantes para realizar la predicción, la primera corresponde una aproximación completa, que es O(n2 m2) en el tiempo y O(nm) en el espacio al restringir el tamaño de los bucles internos y donde n y m son las longitudes de las secuencias de RNA que interactúan (n $>$ m), y la segunda que es una simplificación heurística de la aproximación completa, que tiene una complejidad en tiempo de O(nm) y un espacio de complejidad O(nm), donde m = max {m, L 3} y L es el tamaño de la secuencia de la ventana en el que tanto el $_m$RNA diana y la sRNA se pliegan.

\subsubsection{RNAup}
\par \emph{RNAup~\cite{rnaup}} calcula la termodinámica de las interacciones RNA-RNA, por descomposición de las uniones a
través de dos etapas. La termodinámica de tales interacciones RNA-RNA puede ser entendida como la suma de dos contribuciones de energía: la energía necesaria para ``abrir'' el sitio de unión, y la energía obtenida de la hibridación.

\par En primer lugar se calcula la probabilidad de que un potencial sitio de unión siga siendo desapareado, lo que es equivalente a calcular la energía libre necesaria para romper las uniones. En segundo lugar, este cálculo se combina con la energía de interacción para obtener la energía total de unión.

\par Proporciona dos modos: por defecto calcula accesibilidad (en términos de energía libre para romper uniones, de longitud 4) y muestra la región de mayor accesibilidad y la energía libre necesaria para su apertura. En modo interacción, se calcula la interacción entre dos RNA. Este modo se activa si la entrada se compone de dos secuencias concatenadas por un ``\&'' o si se emplea el flag -X[pf] o -b.

\par  La energía libre ``dG'' se divide en sus componentes ``dGint'' (energía de interacción) y ``dGu\_l'' (energía de apertura).                      

\par \emph{RNAup} esta incluido en el paquete de Vienna RNA~\cite{vienna}}. La última versión corresponde a la 2.0.7 y esta disponible para su descarga en \url{http://www.tbi.univie.ac.at/~ivo/RNA/}.

\subsubsection{RNACofold}
\par \emph{RNACofold} funciona como \emph{RNAfold}, pero permite especificar dos secuencias de RNA que formará una estructura de dímero\footnote{Complejo macromolecular formado por dos macromoléculas, como proteínas y ácidos nucleicos, usualmente mediante enlaces no covalentes.}.             

\par Las secuencias de RNA se leen por entrada estándar, cada línea de entrada corresponde a una secuencia, a excepción de las líneas que comienzan con ``>'', que contiene el nombre de la siguiente secuencia. Para calcular la estructura híbrida de dos moléculas, las dos secuencias deben ser concatenadas usando el carácter ``\&'' como separador. 
            
\par Al igual que RNAup, \emph{RNAcofold} esta incluido en el paquete de Vienna RNA~\cite{vienna}. Tiene orden O($n^{3}$)
    
\subsubsection{RNADuplex}
\par Básicamente lee dos secuencias de RNA por entrada estándar o desde un archivo y calcula las estructuras secundarias óptimas y subóptimas para su hibridación. El cálculo se simplifica al permitir sólo pares de bases inter-moleculares (corresponde a un caso específico de \emph{RNAcofold}). 

\par La estructura calculada óptima y subóptima se retornan por salida estándar, una por línea. Cada una de ellas consiste en: la estructura de soporte en formato de punto con una ``y'' que separan las dos hebras, el rango de la estructura en las dos secuencias en el formato \emph{``from,to : from,to''}, la energía de estructura dúplex en kcal/mol.

\par El formato es especialmente útil para el cálculo de la estructura híbrida entre una secuencia pequeña y una secuencia larga. Al igual que los dos anteriores, \emph{RNAduplex} esta incluido en el paquete de Vienna RNA~\cite{vienna}.         

\subsubsection{RNAHybrid}
\par \emph{RNAHybrid~\cite{rnahybrid}} es una herramienta para encontrar hibridaciones de mínimo de energía libre entre una
secuencia larga (target) y un RNA corto (query). La hibridación se lleva a cabo de la siguiente manera: la secuencia corta se hibrida con las mejores partes de fijación del target.

\par \emph{RNAHybrid} puede obtenerse desde \url{http://bibiserv.techfak.uni-bielefeld.de/download/tools/rnahybrid.html}.  Actualmente, puede utilizarse como servicio web desde \url{http://alk.ibms.sinica.edu.tw/cgi-bin/RNAhybrid/RNAhybrid.cgi}.
Tiene orden O(nm).
