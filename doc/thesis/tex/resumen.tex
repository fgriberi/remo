\begin{center} \LARGE \textbf{Resumen} \end{center}

\par El presente proyecto encuadra en una de las áreas de la Biología, conocida como Virología,
la cual estudia la acción de los virus en el organismo y cómo pueden ser aplicados ciertos fármacos para su tratamiento. 

\par El código genético está organizado en tripletes o codones (conjunto de tres nucleótidos que determinan un aminoácido). Al menos uno de estos tres nucleótidos puede ser distinto y sin embargo codificar para el mismo aminoácido, por lo cual se dice que el código genético esta \emph{degenerado}. Los tripletes que codifican para un mismo aminoácido se denominan \emph{sinónimos}.

\par Se conoce que cada especie tiene un uso de codones sinónimos característico que le permite optimizar la síntesis proteica. Para algunos organismos patógenos intracelulares, como los virus, existe una divergencia entre el uso de codones sinónimos utilizados por el virus y el huésped correspondiente. El origen de esa divergencia no está suficientemente esclarecido. 

\par En el presente trabajo se estudiará si la divergencia en el uso de codones sinónimos entre virus y huésped contribuye a disminuir la interferencia de los microRNA en la expresión de los RNA mensajeros de origen viral. De esa manera se contribuirá a comprender mejor la relación virus-huésped y la evolución viral.

\par Se desarrolló un software que realiza diversas comparaciones masivas entre secuencias de $_m$$_i$RNA y $_m$RNA, además de comparaciones a nivel estructural entre $_m$RNA. El software sigue una implementación distribuida, a través de la utilización de varios procesos que cooperan para realizar las tareas. por otra parte, se implementaron diversas librerías con funcionalidades anexas, que se acoplan con otros paquetes de software relacionados con el área de estudio.

\vskip .5cm
\textbf{Palabras claves:} $_m$$_i$RNA, $_m$RNA, aminoácidos, nucléotidos, FuD, Remo, fideo.

