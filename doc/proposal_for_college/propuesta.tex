\documentclass[12pt,a4paper]{article}

\usepackage[utf8]{inputenc}
\usepackage[spanish]{babel}
\usepackage{xcolor}
\usepackage[pdftex]{graphicx}

\begin{document} 

\title{Propuesta de Trabajo Final Para la Carrera de Licenciatura en Ciencias de la Computación}
		\vskip 2cm
\author{Franco G. Riberi}
		\vskip 2cm
\date{\today} 	

\maketitle

\noindent \textbf{Título:} Estudio de la relación entre divergencia en el uso de codones \\
sinónimos entre virus y huésped y presencia de microRNA. \\
\vskip 0.01cm
\noindent \textbf{Director:} Lic. Laura Tardivo.\\ 
\vskip 0.01cm
\noindent \textbf{Co-director:}  Daniel Gutson.\\
\vskip 0.01cm
\noindent \textbf{Co-director:} Lic. Guillermo Biset.\\
\vskip 0.01cm
\noindent \textbf{Tesista:} Ac. Franco Gaspar Riberi. \\

\section{Descripción del problema}

\par La bioinformática es una disciplina dedicada al análisis de elementos biológicos utilizando la
informática como herramienta principal para generar simulaciones, probar teorías,  realizar cálculos más complejos, etcétera. En particular una de las áreas estudiada es la virología, la cual estudia la acción de un virus y cómo pueden ser aplicados ciertos fármacos para su tratamiento. 

\par Este proyecto nace como sugerencia de \textbf{\textit{FuDePAN}}\footnote{Fundación para el desarrollo de la programación en Ácidos Nucleícos. \textcolor{blue}{http://www.fudepan.org.ar/}}. El principal objetivo es contrastar una teoría postulada y encomendada por el Dr. Roberto Daniel Rabinovich\footnote{Miembro del INBIRS(Instituto Biomedico en Retrovirus y SIDA). Anteriormente CNRS, Centro Nacional de Referencia para el SIDA(Departamento Microbiología UBA). Profesor Titular de Virología (Departamento Ciencias Biológicas, CAECE). Colaborador de \textbf{\textit{FuDePAN}.}}, que involucra principalmente la molécula de ARN\footnote{Ácido Ribonucleico. Es un tipo de ácido nucleico compuesta por nucléotidos esencial para la vida.}.

\par Para la comprensión de la hipótesis se deben tener en cuenta algunos hechos referidos al código genético y los microRNA.
\begin{itemize}
	\item El código está organizado en tripletes o codones: conjunto de tres nucleótidos que determinan un aminoácido.
	\item El código genético es degenerado\footnote{Implica que al menos uno de los tres nucleótidos (en general el último) puede ser distinto y sin 				embargo codificar para el mismo aminoácido.}: existen más tripletes o codones que aminoácidos, de forma que un determinado aminoácido puede 				estar codificado por más de un triplete. Esos tripletes son conocidos como sinónimos. 
	\item En cada especie, se ha seleccionado una proporción de uso de esos codones que guarda relación con la proporción de RNA de transferencia  				correspondiente de manera de optimizar la síntesis proteica.
	\item Para algunos patógenos intracelulares como los virus, existe una divergencia entre el uso de codones sinónimos [12] utilizado por el virus y el 			huésped	correspondiente. El origen de esa divergencia no esta suficientemente esclarecido.
	\item Un microARN [10][11] ($_m$$_i$ARN o $_m$$_i$RNA por sus siglas en inglés) es un ARN monocatenario, de una longitud de entre 21 y 25 nucleótidos, 							y que tiene la capacidad de regular la expresión de otros genes mediante diversos procesos, utilizando para ello la ruta de 						ribointerferencia\footnote{También conocido como RNA$_i$ por el acrónico del inglés RNA interference. Corresponde a un mecanismo 							de silenciamiento post-transcripcional de genes específicos, ejercido por moléculas de ARN que, siendo complementarias a un ARN 						mensajero, conducen habitualmente a la degradación de éste.}. Se encuentran codificados en el genoma y juegan un papel importante 							en la regulación de la expresión proteica, en la embriogénesis\footnote{Proceso que se inicia tras la fertilización de los gametos 							para dar lugar al embrión.}, procesos cancerosos e infecciones virales. La generación de los microRNA puede variar según el 						órgano o temporalmente. 
\end{itemize}

\par En este trabajo se estudiará si la divergencia en el uso de codones sinónimos entre virus y huésped contribuye a disminuir la interferencia de los $_m$$_i$RNA en la expresión de los  RNA mensajeros de origen viral. De esa manera se contribuirá a comprender mejor la relación virus-huésped y la evolución viral.
\par Estos estudios tienen también una importancia potencial en la comprensión de la patogenia viral y en el desarrollo de herramientas terapéuticas como la terapia génica.

\par Formalmente, dado un conjunto de small-RNA$_s$\footnote{Moléculas de RNA muy pequeñas, dentro de las cuales se encuentra la $_m$$_i$RNA, $_s$$_i$RNA, entre otras.} y una colección de RNA$_m$ de un determinado virus, dado que hay un RNA$_m$ por cada gen, determinar si existen small-RNA$_s$ que se van a hibridar al RNA$_m$. Contabilizar la cantidad de small-RNA$_s$ que se hibridarían a la secuencia original o a la secuencia complementaria. De manera similar, realizar el mismo estudio con el genoma humanizado\footnote{La humanización refiere al reemplazo de nucleótidos en los tripletes que forman la secuencia de RNA. Se acerca a la proporción de uso de codones utilizado en el humano.} contabilizando la cantidad de small-RNA$_s$ que se hibridan. 
\par Que se hibride o no un small-RNA$_s$ a un determinado RNA$_m$ involucra distintas reglas, siendo la más importante la complementariedad de bases. Pero además existen otras, como la presencia de determinadas bases, proporción de uniones GC y motivos específicos.

\par Para tener soporte de la ejecución distribuída de la implementación se utilizará el framework \textbf{\textit{FuD}}\footnote{\textbf{\textit{FuDePAN}} Ubiquitous Distribution[1]. \textcolor{blue}{http://code.google.com/p/fud/}} sin tener que desarrollar específicamente esta funcionalidad y sin perder de vista el problema original. Cabe aclarar que el presente desarrollo involucrará un gran número de cálculos dada la gran cantidad de datos a manipular. 

\section{Recursos}

\par Para la obtención del material bibliográfico se cuenta con la biblioteca de la Universidad Nacional de Río Cuarto en conjunto con la amplia cantidad de información que se puede encontrar en Internet y la bibliografía disponible en el Departamento de Computación mediante suscripciones a publicaciones científicas.

\par Además de contar con todo el material bibliográfico, se contará con un equipo interdisciplinar de trabajo, conformado por profesionales de las diferentes áreas involucradas, tanto Biología como Computación. 

\par El sistema operativo a utilizar será GNU/Linux y el producto final deberá ser software libre (bajo licencia GPL [5], versión 3 o superior).

\section{Resultados esperados}
Se espera obtener, para distintos virus, tablas comparativas de posible hibridación entre microRNA presente en el huésped humano y el RNA presente en la naturaleza y el genoma viral humanizado. Además se espera un gráfico con resúmenes estadísticos a través del cual se podrá inferir, según la tendencia de las curvas, si la divergencia en el uso de codones entre virus-huésped y la presencia de microRNA pueden estar relacionados.


\section*{Referencias}
[1] G. Biset, D. Gutson, and M. Arroyo, “A framework for small distributed projects and a protein clusterer application”, 2009. \\

[2] G. Biset, D. Gutson, and M. Arroyo, “Fud: Design and implementation of a framework for small distributed applications”, 2009. \\

[3] B. Meyer, “Object-Oriented Software Construction”, Second Edition, Santa Barbara: Prentice Hall Professional Technical Reference, 1997. \\

[4] G. Booch, J. Rumbaugh, and I. Jacobson, “Unified Modeling Language User Guide”, Second Edition, 2005. \\

[5] GNU General Public License. \textcolor{blue}{http://www.gnu.org/licenses/} \\

[6] H. Curtis, N. Sue Barnes, A. Schnek and G. Flores, “Biología”, Editorial Médica Panamericana S.A, 2006, ISBN: 950-06-0423-X. \\

[7] B. Pierce, “Genética. Un enfoque conceptual”, Tercera Edición, Editorial médica panamericana S.A, ISBN: 978-84-9835-216-0. \\

[8] A. Blanco, “Química Biológica”, Séptima Edición, Editorial El Ateneo. \\

[9] Haruhiko Siomi and Mikiko C. Siomi, “On the road to reading the RNA-interference code”. \\

[10] Vinay S. Mahajan, Adam Drake and Jianzhu Chen, “Virus-specific host miRNAs: antiviral defenses or promoters of persistent infection?”. \\

[11] Man Lung YEUNG, Yamina BENNASSER, Shu Yun LE and Kuan Teh JEANG, “siRNA, miRNA and HIV: promises and challenges”. \\

[12] Gareth M. Jenkins and Edward C. Holmes, “The extent of codon usage bias in human RNA viruses and its evolutionary origin”, 2003. \\

[13] Comeron JM and Aguadé M. “An evaluation of measures of synonymous codon usage bias”, 1998. \\
\end{document}

